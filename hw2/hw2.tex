%%%%%%%%%%%%%%%%%%%%%%%%%%%%%%%%%%%%%%%%%
% Structured General Purpose Assignment
% LaTeX Template
%
% This template has been downloaded from:
% http://www.latextemplates.com
%
% Original author:
% Ted Pavlic (http://www.tedpavlic.com)
%
% Note:
% The \lipsum[#] commands throughout this template generate dummy text
% to fill the template out. These commands should all be removed when 
% writing assignment content.
%
%%%%%%%%%%%%%%%%%%%%%%%%%%%%%%%%%%%%%%%%%

%----------------------------------------------------------------------------------------
%	PACKAGES AND OTHER DOCUMENT CONFIGURATIONS
%----------------------------------------------------------------------------------------

\documentclass{article}

\usepackage{fancyhdr} % Required for custom headers
\usepackage{lastpage} % Required to determine the last page for the footer
\usepackage{extramarks} % Required for headers and footers
\usepackage{graphicx} % Required to insert images
\usepackage{lipsum} % Used for inserting dummy 'Lorem ipsum' text into the template
\usepackage{listings}
\usepackage{color}
\usepackage{amsmath}

\definecolor{dkgreen}{rgb}{0,0.6,0}
\definecolor{gray}{rgb}{0.5,0.5,0.5}
\definecolor{mauve}{rgb}{0.58,0,0.82}

\lstset{frame=tb,
  language=Python,
  aboveskip=3mm,
  belowskip=3mm,
  showstringspaces=false,
  columns=flexible,
  basicstyle={\small\ttfamily},
  numbers=none,
  numberstyle=\tiny\color{gray},
  keywordstyle=\color{blue},
  commentstyle=\color{dkgreen},
  stringstyle=\color{mauve},
  breaklines=true,
  breakatwhitespace=true
  tabsize=3
}

% Margins
\topmargin=-0.45in
\evensidemargin=0in
\oddsidemargin=0in
\textwidth=6.5in
\textheight=9.0in
\headsep=0.25in 

\linespread{1.1} % Line spacing

% Set up the header and footer
\pagestyle{fancy}
\lhead{\hmwkAuthorName} % Top left header
\chead{\hmwkClass\ (\hmwkClassInstructor\ \hmwkClassTime): \hmwkTitle} % Top center header
\rhead{\firstxmark} % Top right header
\lfoot{\lastxmark} % Bottom left footer
\cfoot{} % Bottom center footer
\rfoot{Page\ \thepage\ of\ \pageref{LastPage}} % Bottom right footer
\renewcommand\headrulewidth{0.4pt} % Size of the header rule
\renewcommand\footrulewidth{0.4pt} % Size of the footer rule

\setlength\parindent{0pt} % Removes all indentation from paragraphs

%----------------------------------------------------------------------------------------
%	DOCUMENT STRUCTURE COMMANDS
%	Skip this unless you know what you're doing
%----------------------------------------------------------------------------------------

% Header and footer for when a page split occurs within a problem environment
\newcommand{\enterProblemHeader}[1]{
\nobreak\extramarks{#1}{#1 continued on next page\ldots}\nobreak
\nobreak\extramarks{#1 (continued)}{#1 continued on next page\ldots}\nobreak
}

% Header and footer for when a page split occurs between problem environments
\newcommand{\exitProblemHeader}[1]{
\nobreak\extramarks{#1 (continued)}{#1 continued on next page\ldots}\nobreak
\nobreak\extramarks{#1}{}\nobreak
}

\setcounter{secnumdepth}{0} % Removes default section numbers
\newcounter{homeworkProblemCounter} % Creates a counter to keep track of the number of problems

\newcommand{\homeworkProblemName}{}
\newenvironment{homeworkProblem}[1][Problem \arabic{homeworkProblemCounter}]{ % Makes a new environment called homeworkProblem which takes 1 argument (custom name) but the default is "Problem #"
\stepcounter{homeworkProblemCounter} % Increase counter for number of problems
\renewcommand{\homeworkProblemName}{#1} % Assign \homeworkProblemName the name of the problem
\section{\homeworkProblemName} % Make a section in the document with the custom problem count
\enterProblemHeader{\homeworkProblemName} % Header and footer within the environment
}{
\exitProblemHeader{\homeworkProblemName} % Header and footer after the environment
}

\newcommand{\problemAnswer}[1]{ % Defines the problem answer command with the content as the only argument
\noindent\framebox[\columnwidth][c]{\begin{minipage}{0.98\columnwidth}#1\end{minipage}} % Makes the box around the problem answer and puts the content inside
}

\newcommand{\homeworkSectionName}{}
\newenvironment{homeworkSection}[1]{ % New environment for sections within homework problems, takes 1 argument - the name of the section
\renewcommand{\homeworkSectionName}{#1} % Assign \homeworkSectionName to the name of the section from the environment argument
\subsection{\homeworkSectionName} % Make a subsection with the custom name of the subsection
\enterProblemHeader{\homeworkProblemName\ [\homeworkSectionName]} % Header and footer within the environment
}{
\enterProblemHeader{\homeworkProblemName} % Header and footer after the environment
}
   
%----------------------------------------------------------------------------------------
%	NAME AND CLASS SECTION
%----------------------------------------------------------------------------------------

\newcommand{\hmwkTitle}{Homework 2} % Assignment title
\newcommand{\hmwkDueDate}{Aug 28,\ 2014} % Due date
\newcommand{\hmwkClass}{Numerical Linear Algebra} % Course/class
\newcommand{\hmwkClassTime}{6:00 pm} % Class/lecture time
\newcommand{\hmwkClassInstructor}{Lecture time:} % Teacher/lecturer
\newcommand{\hmwkAuthorName}{Weiyi Chen} % Your name

%----------------------------------------------------------------------------------------
%	TITLE PAGE
%----------------------------------------------------------------------------------------

\title{
\vspace{2in}
\textmd{\textbf{\hmwkClass:\ \hmwkTitle}}\\
\normalsize\vspace{0.1in}\small{Due\ on\ \hmwkDueDate}\\
\vspace{0.1in}\large{\textit{\hmwkClassInstructor\ \hmwkClassTime}}
\vspace{3in}
}

\author{\textbf{\hmwkAuthorName}}
\date{} % Insert date here if you want it to appear below your name

%----------------------------------------------------------------------------------------

\begin{document}

\maketitle

%----------------------------------------------------------------------------------------
%	TABLE OF CONTENTS
%----------------------------------------------------------------------------------------

%\setcounter{tocdepth}{1} % Uncomment this line if you don't want subsections listed in the ToC

%\newpage
%\tableofcontents
\newpage

%----------------------------------------------------------------------------------------
%   PROBLEM 1
%----------------------------------------------------------------------------------------

\begin{homeworkProblem}
  \begin{homeworkSection}{(i)}
    Let $det(\lambda I - A) = 0$, the eigenvalues of the matrix $A$ are
    \begin{equation}
      \lambda_1 = -2, \lambda_2 = 3
    \end{equation}
    The (normalized) corresponded eigenvectors are
    \begin{equation}
      v_1 = [-0.894427, 0.447214], v_2 = [-0.447214, -0.894427]
    \end{equation}
  \end{homeworkSection}
  \begin{homeworkSection}{(ii)}
    The diagonal form of $A$ is 
    \begin{equation}
      A = V\Lambda V^{-1}
    \end{equation}
    where
    \begin{equation}
      V = col(v_1, v_2), \Lambda = diag(\lambda_1, \lambda_2)
    \end{equation}
    In other words,
    \begin{equation}
      A = \left(\begin{array}{cc} 
      -0.894427 & -0.447214 \\
      0.447214 & -0.894427
      \end{array}\right)
      \left(\begin{array}{cc}
      -2 & 0 \\
      0 & 3 \\
      \end{array}\right)
      \left(\begin{array}{cc}
      -0.894427 & 0.447214 \\
      -0.447214 & -0.894427
      \end{array}\right)
    \end{equation}
  \end{homeworkSection}
  \begin{homeworkSection}{(iii)}
    \begin{equation}
      A^{12} = (V\Lambda V^{-1})^{12} = V \Lambda^{12} V^{-1} = \left(\begin{array}{cc}
      109565 & 210938 \\
      210938 & 425972
      \end{array}\right)
    \end{equation}
  \end{homeworkSection}
\end{homeworkProblem}

%----------------------------------------------------------------------------------------
%   PROBLEM 2
%----------------------------------------------------------------------------------------

\begin{homeworkProblem}
   Let $v \in C$ be an eigenvector of A corresponding to the eigenvalue $\lambda \in C$, then I will show that $A\overline v = \overline\lambda \overline v$. Taking the complex conjugate of $Av = \lambda v$, we obtain
   \begin{equation}
     A\overline v = \overline{Av} = \overline{\lambda v} = \overline \lambda \overline v
   \end{equation}
   To see this, we repeatedly use the fact that $\overline{w + z} = \overline w + \overline z$ and $\overline{wz} = \overline w \overline z$ holds for all $w, z \in C$. For the right hand side, we have
   \begin{equation}
     \overline{\lambda v} = \left[\begin{array}{c} 
     \overline{\lambda v_1} \\
     \dots \\
     \overline{\lambda v_n}
     \end{array}\right] = \left[\begin{array}{c} 
     \overline\lambda \overline v_1 \\
     \dots \\
     \overline\lambda \overline v_n
     \end{array}\right]
     = \overline \lambda \overline v
   \end{equation}
   For the left hand side, we have
   \begin{equation}
     \overline{Av} = \left[\begin{array}{c} 
     \overline{a_{11}v_1 + \dots + a_{1n}v_{n}} \\
     \dots \\
     \overline{a_{n1}v_1 + \dots + a_{nn}v_{n}}
     \end{array}\right] = \left[\begin{array}{c} 
     \overline{a_{11}v_1} + \dots + \overline{a_{1n}v_{n}} \\
     \dots \\
     \overline{a_{n1}v_1} + \dots + \overline{a_{nn}v_{n}}
     \end{array}\right] =
     \left[\begin{array}{c} 
     a_{11}\overline{v_1} + \dots + a_{1n}\overline{v_{n}} \\
     \dots \\
     a_{n1}\overline{v_1} + \dots + a_{nn}\overline{v_{n}}
     \end{array}\right] = A\overline v
   \end{equation}
   where we used the fact that all enties of A are real (so $a_{ij} = a_{ij}$). In conclude, $\overline \lambda$ is also an eigenvalue of $A$.
\end{homeworkProblem}

%----------------------------------------------------------------------------------------
%   PROBLEM 3
%----------------------------------------------------------------------------------------

\begin{homeworkProblem}
  \begin{homeworkSection}{(i)}
    Given the eigenvalues $\lambda_i$ and corresponded eigenvectors $v_i$,
    \begin{equation}
      A^2v_i = A(Av_i) = A(\lambda_i v_i) = \lambda_i Av = \lambda_i^2 v_i
    \end{equation}
    therefore the eigenvalues of $A$ are $\lambda_i^2$ and the eigenvectors of $A$ are $v_i$, for $i = 1:n$.
  \end{homeworkSection}
  \begin{homeworkSection}{(ii)}
    Since $A$ is symmetric, according to part(i), $A^2$ can be written as the diagonal form as well. \\
    Suppose the diagonal form of $A^2$ is
    \begin{equation}
      A^2 = V \Phi V^{-1}
    \end{equation}
    where $V = col(w_i)$, and $\Phi = diag(\phi_i)$. Then
    \begin{equation}
      A = V \Phi^{1/2} V^{-1}
    \end{equation}
    since $A^2 = (V \Phi^{1/2} V^{-1})^2 = V \Phi V^{-1}$. Therefore the eigenvectors of $A$ is $\phi_i^{1/2}$ for $i=1:n$, note that $\phi_i^{1/2} \in C$. And same as part(i), the eigenvectors of $A$ are also $w_i$ for $i=1:n$.
  \end{homeworkSection}
\end{homeworkProblem}

%----------------------------------------------------------------------------------------
%   PROBLEM 4
%----------------------------------------------------------------------------------------

\begin{homeworkProblem}
  Since $P_A(t)$ is the characteristic polynomial associated to $A$, then when $t = \lambda$ as the eigenvalue of $A$,
  \begin{equation}
    P_A(\lambda) = \lambda^2 + (a+d)\lambda + (ad-bc) = 0
  \end{equation}
  According to the property of eigenvalue $Av = \lambda v$ where $v$ is the corresponded eigenvector,
  \begin{align}
    P_A(A) \cdot v &= A^2v + (a+d)Av + (ad-bc)I \\
    &= A(\lambda v) + (a+d)\lambda Iv + (ad-bc)I \\
    &= \lambda^2Iv + (a+d)\lambda Iv + (ad-bc)Iv\\
    &= P_A(\lambda)Iv = 0
  \end{align}
\end{homeworkProblem}

%----------------------------------------------------------------------------------------
%   PROBLEM 5
%----------------------------------------------------------------------------------------

\begin{homeworkProblem}
  For every $j = 1:n$, let
  \begin{equation}
    R_j = \sum_{k\neq j} |A(j,k)|
  \end{equation}
  and denote by $D_j$ the disc of center $A(j,j)$ and radius $R_j$, i.e.,
  \begin{equation}
    D_j = {z \in C \text{ such that } |z-A(j,j)| \le R_j}
  \end{equation}
  where $D_j$ is called a Gershgorin disk corresponding to the matrix $A$. A more general form of Gershgorin's theorem states that, if a Gershgorin disk $D_i$ is disjoint from the union of the other $n-1$ Gershgorin disks of $A$, then exactly one eigenvalue of the matrix $A$ is in the disk $D_i$. \\
  For our problem, all Gershgorin disks are disjoint from each other, because (easy to verify on axis)
  \begin{align}
    R_1 &= 0.003, D_1 = {z \in C \text{ such that } |z-2| \le R_1} \\
    R_2 &= 0.0013, D_2 = {z \in C \text{ such that } |z+1.25| \le R_2} \\
    R_3 &= 0.0025, D_3 = {z \in C \text{ such that } |z-3| \le R_3} \\
    R_4 &= 0.0021, D_4 = {z \in C \text{ such that } |z+2.5| \le R_4}
  \end{align}
  According to problem 2, the complex conjugate of eigenvalue is still the eigenvalue of $A$ and the complex conjurate is symmetric to its original value with respect to the axis. In other words, if the eigenvalue is in the disk $D_i$, its complex conjurate is also in the disk $D_i$. \\
  However according to Gershgorin's theorem, exactly one eigenvalue of the matrix $A$ is in the disk $D_i$ for $i=1:4$, which implies that all eigenvalues have to be on the axis (the complex conjurate is itself), so all the eigenvalues of the matrix are real numbers. \\
  The estimates for the eigenvalues are as follows according to the disks
  \begin{equation}
    \lambda_1 = 2 \pm 0.003, \lambda_2 = -1.25 \pm 0.0013, \lambda_3 = 3 \pm 0.0025, \lambda_4 = -2.5 \pm 0.0021
  \end{equation}
  where all the disk radius are within $0.005$ accuracy.
\end{homeworkProblem}

%----------------------------------------------------------------------------------------
%   PROBLEM 6
%----------------------------------------------------------------------------------------

\begin{homeworkProblem}
  \begin{homeworkSection}{(i)}
    The payoff matrix $M$ is
    \begin{equation}
      M = \left(\begin{array} {cc} 
        e^{r\delta t} & e^{r\delta t} \\
        S_u & S_d
      \end{array} \right) = \left(\begin{array} {cc} 
        e^{0.015} & e^{0.015} \\
        60 & 45
      \end{array} \right)
    \end{equation}
  \end{homeworkSection}
  \begin{homeworkSection}{(ii)}
    The market is complete, since the matrix $M$ is nonsingular with $e^{0.015}\times(45 - 60) \neq 0$.
  \end{homeworkSection}
  \begin{homeworkSection}{(iii)}
    Let $s$ be the price vector of the replicable security, then
    \begin{equation}
      s = (0, 5)^t
    \end{equation}
    To replicate, let $\Theta$ be the positions vector of the replicating portfolio. Then,
    \begin{equation}
      s = \Theta^t M \Rightarrow \Theta^t = s M^{-1} = (19.702239,-0.333333)
    \end{equation}
    which is to long $\$19.702239$ cash and short $-\$0.333333$ the asset.
  \end{homeworkSection}
\end{homeworkProblem}

%----------------------------------------------------------------------------------------
%   PROBLEM 7
%----------------------------------------------------------------------------------------

\begin{homeworkProblem}
  \begin{homeworkSection}{(i)}
    The necessary and sufficient condition for the one period market model to be complete is that, the row rank of the payoff matrix is equal to $n$, the number of states. In our problem, the payoff matrix is
    \begin{equation}
      M = \left(\begin{array}{cc}
        FV_1 & FV_2 \\
        uS_0 & dS_0
      \end{array}\right)
    \end{equation}
    The row rank of the payoff matrix is equal to $n=2$, if and only if $det(M)\neq 0$, that is 
    \begin{equation}
      dS_0 \cdot FV_1 \neq uS_0 \cdot FV_2
    \end{equation}
    In other words, the required necessary and sufficient condition is $d \cdot FV_1 \neq u \cdot FV_2$.
  \end{homeworkSection}
  \begin{homeworkSection}{(ii)}
    The price vector of the securities at time $t_0$ is
    \begin{equation}
      S_{t_0} = (1, S_0)^t
    \end{equation}
    To decide whether this market model is arbitrage-free, we must solve the linear system $MQ = S_{t_0}$ and check whether all the entries of the vector $Q$ are positive, as stated
    \begin{align}
      Q &= M^{-1}S_{t_0} = \frac{(d-FV_2, -u+FV_1)^t}{d\cdot FV_1-u\cdot FV_2} 
    \end{align}
    To make all the entries are positive, when $d\cdot FV_1-u\cdot FV_2 > 0$, it requires $d-FV_2>0$ and $-u+FV_1>0$, i.e.,
    \begin{equation}
      \frac{d}{FV_2}>1, \frac{u}{FV_1} < 1
    \end{equation}
    when $d\cdot FV_1-u\cdot FV_2 < 0$, we have
    \begin{equation}
      \frac{d}{FV_2}<1, \frac{u}{FV_1} > 1
    \end{equation}
    Therefore in other words, at least one of $\frac{d}{FV_2}$ and $\frac{u}{FV_1}$ should be large than $1$ and the other should be less than $1$, i.e.,
    \begin{equation}
      \min(\frac{u}{FV_1},\frac{d}{FV_2}) < 1 < \max(\frac{u}{FV_1},\frac{d}{FV_2})
    \end{equation}
  \end{homeworkSection}
  \begin{homeworkSection}{(iii)}
    In the classical one period binomial model, we are given
    \begin{equation}
      FV_1 = FV_2 = e^{r\delta t},d<u
    \end{equation}
    then
    \begin{align}
      \min(\frac{u}{FV_1},\frac{d}{FV_2}) &= \frac{d}{e^{r\delta t}} \\
      \max(\frac{u}{FV_1},\frac{d}{FV_2}) &= \frac{u}{e^{r\delta t}}
    \end{align}
    We are able to derive the no-arbitrage condition,
    \begin{equation}
      d < e^{r\delta t} < u
    \end{equation}
  \end{homeworkSection}
\end{homeworkProblem}

%----------------------------------------------------------------------------------------
%   PROBLEM 8
%----------------------------------------------------------------------------------------

\begin{homeworkProblem}
  The risk-neutral probabilities associated to each of the two states of the market at time $\tau$ are
  \begin{align}
    p_{RN}(1) &= e^{r\delta t}Q^1 = \frac{e^{r \delta t}S_0-S_d}{S_u-S_d} \\
    p_{RN}(2) &= e^{r\delta t}Q^2 = \frac{S_u-e^{r \delta t}S_0}{S_u-S_d}
  \end{align}
  The value at time $t_0$ of a derivative security with payoffs at time $\tau$ equal to $V_\tau(1)$, if the "up" state occurs, and equal to $V_\tau(2)$, if the "down" state occurs, is
  \begin{equation}
    V_{t_0} = e^{-r\delta t}(p_{RN}(1)V_\tau(1) + p_{RN}(2)V_\tau(2))
  \end{equation}
  Given the ATM put price as $\$4$, that is, $V_\tau(1) = 0, V_\tau(2) = 10, V_{t_0}=4$, we are able to derive
  \begin{equation}
    e^{-r \delta t} = \frac{29}{30}
  \end{equation}
  Then the ATM call price is (given $V_\tau(1) = 10, V_\tau(2) = 0$)
  \begin{equation}
    C_{t_0} = e^{-r\delta t}p_{RN}(1)V_\tau(1) = (5-4x)\times5 = 5.666667
  \end{equation}
  The value of the three months ATM call is $\$5.666667$.
\end{homeworkProblem}

%----------------------------------------------------------------------------------------
%   PROBLEM 9
%----------------------------------------------------------------------------------------

\begin{homeworkProblem}
  \begin{homeworkSection}{(i)}
    Let $X=MZ$, then the correlation matrix of $X$ is
    \begin{equation}
      \Omega_X = M\Omega_ZM^t = MM^t
    \end{equation}
    Apply Cholesky decomposition to the given matrix as of $\Omega_X$, we derive
    \begin{equation}
      M = \left(\begin{array} {ccc} 
        1 & 0 & 0 \\
        -0.3 & 0.9539392 & 0 \\
        -0.2 & 0.46124533 & 0.86443782
      \end{array} \right)
    \end{equation}
    Therefore the three normal random variables are
    \begin{align}
      X_1 &= Z_1 \\
      X_2 &= -0.3Z_1 + 0.953939Z_2 \\
      X_3 &= -0.2Z_1 + 0.461245Z_2 + 0.864438Z_3
    \end{align}
  \end{homeworkSection}
  \begin{homeworkSection}{(ii)}
    Let $X=MZ$, then the covariance matrix of $X$ is
    \begin{equation}
      \Sigma_X = M\Sigma_ZM^t = MM^t
    \end{equation}
    Apply Cholesky decomposition to the given matrix as of $\Sigma_X$, we derive
    \begin{equation}
      M = \left(\begin{array} {ccc} 
        2 & 0 & 0 \\
        1.5 & 1.32287566 & 0 \\
        -0.15 & 0.01889822 & 0.98850537
      \end{array} \right)
    \end{equation}
    Therefore the three normal random variables are
    \begin{align}
      X_1 &= 2Z_1 \\
      X_2 &= 1.5Z_1 + 1.322876Z_2 \\
      X_3 &= -0.15Z_1 + 0.018898Z_2 + 0.988505Z_3
    \end{align}
  \end{homeworkSection}
\end{homeworkProblem}

%----------------------------------------------------------------------------------------
%   PROBLEM 10
%----------------------------------------------------------------------------------------

\begin{homeworkProblem}
  \begin{homeworkSection}{(i)}
    The correlation matrix of the random variables are
    \begin{equation}
      \Omega_X = \left(\begin{array}{ccc}
      1 & \rho & 0.3 \\
      \rho & 1 & 0.1 \\
      0.3 & 0.1 & 1
      \end{array} \right)
    \end{equation}
    To make it a correlation matrix, we require it to be symmetric positive semidefinite, that is, all the principle minors should be at least $0$,
    \begin{equation}
      \left| \begin{array} {cc}
      1 & \rho \\
      \rho & 1
      \end{array} \right| = 1 -\rho^2 \ge 0
    \end{equation}
    and
    \begin{equation}
      det(\Omega_X) = det\left(\begin{array}{ccc}
      1 & \rho & 0.3 \\
      \rho & 1 & 0.1 \\
      0.3 & 0.1 & 1
      \end{array} \right) \ge 0
    \end{equation}
    We derive 
    \begin{align}
      -1 \le &\rho \le 1 \\
      -0.919157521173 \le &\rho \le 0.979157521173
    \end{align}
    Therefore the upper bound for $\rho$ is $0.979158$ and the lower bound is $-0.919158$.
  \end{homeworkSection}
  \begin{homeworkSection}{(ii)}
    The correlation matrix of the random variables are
    \begin{equation}
      \Omega_X = \left(\begin{array}{cccc}
      1 & \rho & 0.3 & 0.2 \\
      \rho & 1 & 0.1 & -0.1 \\
      0.3 & 0.1 & 1 & -0.2 \\
      0.2 & -0.1 & -0.2 & 1
      \end{array} \right)
    \end{equation}
    To make it a correlation matrix, we require it to be symmetric positive semidefinite, that is, determinants of all the principle minors should be at least $0$, besides the conditions of part(i) (which are order of 2 and order of 3), we furtherly require another order of 3 and order of 4, i.e.,
    \begin{equation}
      det\left(\begin{array}{ccc}
      1 & \rho & 0.2 \\
      \rho & 1 & -0.1 \\
      0.2 & -0.1 & 1
      \end{array} \right) \ge 0
    \end{equation}
    and
    \begin{equation}
      det(\Omega_X) = det\left(\begin{array}{cccc}
      1 & \rho & 0.3 & 0.2 \\
      \rho & 1 & 0.1 & -0.1 \\
      0.3 & 0.1 & 1 & -0.2 \\
      0.2 & -0.1 & -0.2 & 1
      \end{array} \right) \ge 0
    \end{equation}
    We derive 
    \begin{align}
      -0.919157521173 \le &\rho \le 0.979157521173 \\
      -0.994884608556 \le &\rho \le 0.954884608556 \\
      -0.900286652017 \le &\rho \le 0.916953318684
    \end{align}
    In conclusion, the upper bound for $\rho$ is $0.916953$ and the lower bound is $-0.900287$.
  \end{homeworkSection}
\end{homeworkProblem}

%----------------------------------------------------------------------------------------
%   PROBLEM 11
%----------------------------------------------------------------------------------------

\begin{homeworkProblem}
  To make the given matrix as a correlation matrix, we require the determinants of all its principle minors to be at least $0$, so that it is symmetric positive semidefinite. In this matrix, it is easy to find that principle minors with same order are the same, so we just need to verify $n$ matrix determinants, i.e.,
  \begin{equation}
    det(A_i) = \left|\begin{array}{ccccc} 
    1 & q & \dots & q & q \\
    q & 1 & \dots & q & q \\
    \dots \\
    q & q & \dots & 1 & q \\
    q & q & \dots & q & 1
    \end{array}\right| \ge 0
  \end{equation}
  where the size of $A_i$ is $i$ for $i=1:n$. To derive the formula of this determinant, we need to do some transformation:
  \begin{align}
    det(A_i) &= [(i-1)q+1] \left|\begin{array}{ccccc} 
    1 & 1 & \dots & 1 & 1 \\
    q & 1 & \dots & q & q \\
    \dots \\
    q & q & \dots & 1 & q \\
    q & q & \dots & q & 1
    \end{array}\right| \\
    &= [(i-1)q+1] \left|\begin{array}{ccccc} 
    1 & 1 & \dots & 1 & 1 \\
    0 & 1-q & \dots & 0 & 0 \\
    \dots \\
    0 & 0 & \dots & 1-q & 0 \\
    0 & 0 & \dots & 0 & 1-q
    \end{array}\right| \\
    &= [(i-1)q+1](1-q)^{i-1} > 0
  \end{align}
  Therefore the lower bound restrictions for $q$ are
  \begin{equation}
    q \ge - \frac{1}{i-1} \text{ for }i=1:n
  \end{equation}
  In conclusion, the upper bound is $1$ and the lower bound is $-\frac{1}{n-1}$.
\end{homeworkProblem}

%----------------------------------------------------------------------------------------
%   PROBLEM 12
%----------------------------------------------------------------------------------------

\begin{homeworkProblem}
  \begin{homeworkSection}{(i)}
    The three vectors are linearly independent. We first add the first vector 2 times to the second vector, and subtract 6 times of the first vector from the third vector, we obtain
    \begin{equation}
      \left(\begin{array}{c} 
      0.25 \\
      1 \\
      -0.5 \\
      0 \\
      -1
      \end{array} \right),
      \left(\begin{array}{c} 
      0 \\
      1 \\
      -1.25 \\
      0.5 \\
      -1.25
      \end{array} \right),
      \left(\begin{array}{c} 
      0 \\
      -4.75 \\
      5 \\
      0.75 \\
      7.25
      \end{array} \right),
    \end{equation}
    then add 4 times of the second vector to the third vector, we obtain
    \begin{equation}
      \left(\begin{array}{c} 
      0.25 \\
      1 \\
      -0.5 \\
      0 \\
      -1
      \end{array} \right),
      \left(\begin{array}{c} 
      0 \\
      1 \\
      -1.25 \\
      0.5 \\
      -1.25
      \end{array} \right),
      \left(\begin{array}{c} 
      0 \\
      -0.75 \\
      0 \\
      2.75 \\
      2.25
      \end{array} \right),
    \end{equation}
    Therefore $rank(A) = 3$ where $A$ is the given matrix, the three random variables are linearly independent.
  \end{homeworkSection}
  \begin{homeworkSection}{(ii)}
    The covariance matrix of them is
    \begin{equation}
      \Sigma_X = \frac{1}{N}\bar A \bar A^t = 
      \left(\begin{array}{ccc}
      0.575 & -0.44375 & -0.13125 \\
      -0.44375 & 0.51875 & -0.075 \\
      -0.13125 & -0.075 & 0.20625
      \end{array} \right)
    \end{equation}
    It's easy to verify that
    \begin{equation}
      det(\Sigma_X) = 0
    \end{equation}
    the covariance matrix is singular.
  \end{homeworkSection}
\end{homeworkProblem}

%----------------------------------------------------------------------------------------
%   PROBLEM 13
%----------------------------------------------------------------------------------------

\begin{homeworkProblem}
  \begin{homeworkSection}{(i)}
    The weekly log returns of these eight stocks is (only show the first 10 lines):
    \begin{lstlisting}
        AXP       BAC       JPM      CSCO       HPQ       IBM      INTC      MSFT
0 -0.019510 -0.013307 -0.015963  0.005776 -0.026559  0.022515 -0.001251 -0.037426  
1 -0.004663  0.029505  0.022430 -0.016173  0.024137  0.033869  0.029190  0.033336 
2  0.067113  0.023442  0.021196  0.016584 -0.003239  0.020320 -0.013461  0.036221 
3 -0.030274  0.019978  0.005737 -0.015331 -0.015530  0.008101 -0.012813 -0.021690  
4 -0.005846 -0.031253 -0.017123 -0.024515 -0.008687 -0.011125 -0.014666 -0.018344  
5 -0.026193  0.014259 -0.025669  0.029513 -0.129326  0.013104  0.005053  0.017478 
6 -0.045192 -0.031028 -0.065875 -0.138828  0.012218 -0.019229  0.018723 -0.071512  
7  0.002145 -0.061826 -0.028800 -0.012972 -0.042946 -0.004483 -0.022089  0.023906 
8 -0.036061  0.014611  0.027153 -0.007280 -0.032709 -0.051153 -0.032986 -0.051899  
9 -0.007431 -0.063196 -0.034378 -0.010282 -0.020357 -0.022439 -0.041802 -0.065581   
    \end{lstlisting}
    The monthly log returns of these eight stocks is 
    \begin{lstlisting}
        AXP       BAC       JPM      CSCO       HPQ       IBM      INTC      MSFT
0  0.012666  0.059618  0.033399 -0.009144 -0.021191  0.084785  0.002061  0.010369 
1 -0.101423 -0.104140 -0.107639 -0.142459 -0.170204 -0.031349 -0.022913 -0.059926   
2 -0.016857 -0.010625 -0.038547 -0.008704  0.146026  0.051715 -0.092197 -0.145819  
3 -0.082008 -0.081276  0.030293 -0.001945 -0.174755 -0.060233 -0.044893 -0.028563  
4 -0.055237  0.007105 -0.038873 -0.014212 -0.198941  0.006930 -0.003389 -0.020861  
5 -0.026345  0.025340 -0.074672 -0.045439 -0.147023 -0.058410 -0.020084 -0.027368  
6 -0.027960 -0.164704 -0.067949 -0.038436 -0.102366 -0.007773 -0.052309 -0.003427  
7  0.001263 -0.055425  0.014552 -0.105182  0.063997  0.018825  0.089702  0.061387 
8  0.012369 -0.054127 -0.036658  0.100339  0.208423  0.064304  0.046563  0.041797 
9  0.024984 -0.101458 -0.085881 -0.001079 -0.018138 -0.062639  0.091664  0.035151   
    \end{lstlisting}
  \end{homeworkSection}
  \begin{homeworkSection}{(ii)}
    The sample covariance matrix of weekly log return:
    \begin{lstlisting}
[[ 0.001033  0.00133   0.000992  0.00046   0.00067   0.000449  0.000552  0.000542]
 [ 0.00133   0.0036    0.002139  0.000977  0.001152  0.000857  0.000896  0.000883]
 [ 0.000992  0.002139  0.001949  0.000937  0.001114  0.000684  0.000692  0.000687]
 [ 0.00046   0.000977  0.000937  0.001482  0.000927  0.000589  0.000534  0.000525]
 [ 0.00067   0.001152  0.001114  0.000927  0.003587  0.000872  0.001016  0.000591]
 [ 0.000449  0.000857  0.000684  0.000589  0.000872  0.000865  0.000519  0.000428]
 [ 0.000552  0.000896  0.000692  0.000534  0.001016  0.000519  0.001066  0.000507]
 [ 0.000542  0.000883  0.000687  0.000525  0.000591  0.000428  0.000507  0.000916]]
    \end{lstlisting}
    The sample correlation matrix of weekly log return:
    \begin{lstlisting}
[[ 1.        0.689501  0.699261  0.37207   0.348173  0.474895  0.526059  0.556712]
 [ 0.689501  1.        0.807547  0.423249  0.320555  0.485617  0.457506  0.486391]
 [ 0.699261  0.807547  1.        0.551521  0.42147   0.527166  0.480359  0.514542]
 [ 0.37207   0.423249  0.551521  1.        0.402049  0.5203    0.424725  0.450444]
 [ 0.348173  0.320555  0.42147   0.402049  1.        0.495163  0.519735  0.325845]
 [ 0.474895  0.485617  0.527166  0.5203    0.495163  1.        0.540684  0.481144]
 [ 0.526059  0.457506  0.480359  0.424725  0.519735  0.540684  1.        0.513052]
 [ 0.556712  0.486391  0.514542  0.450444  0.325845  0.481144  0.513052  1.      ]]
    \end{lstlisting}
    The sample covariance matrix of monthly log return:
    \begin{lstlisting}
[[ 0.003105  0.005148  0.00449   0.002934  0.004046  0.001222  0.001832  0.002217]
 [ 0.005148  0.018797  0.011645  0.005865  0.00561   0.002353  0.002242  0.004682]
 [ 0.00449   0.011645  0.011253  0.005719  0.004723  0.002201  0.002362  0.004006]
 [ 0.002934  0.005865  0.005719  0.00765   0.004756  0.001967  0.002488  0.002816]
 [ 0.004046  0.00561   0.004723  0.004756  0.018312  0.003346  0.003729  0.002116]
 [ 0.001222  0.002353  0.002201  0.001967  0.003346  0.001998  0.001216  0.000768]
 [ 0.001832  0.002242  0.002362  0.002488  0.003729  0.001216  0.003887  0.002494]
 [ 0.002217  0.004682  0.004006  0.002816  0.002116  0.000768  0.002494  0.003533]]
    \end{lstlisting}
    The sample correlation matrix of monthly log return:
    \begin{lstlisting}
[[ 1.        0.673816  0.759623  0.602003  0.536523  0.490684  0.527247  0.669284]
 [ 0.673816  1.        0.800678  0.489094  0.302386  0.383878  0.262332  0.574497]
 [ 0.759623  0.800678  1.        0.616332  0.328974  0.464213  0.35722   0.635266]
 [ 0.602003  0.489094  0.616332  1.        0.401785  0.503133  0.456288  0.541684]
 [ 0.536523  0.302386  0.328974  0.401785  1.        0.553181  0.44198   0.263062]
 [ 0.490684  0.383878  0.464213  0.503133  0.553181  1.        0.43652   0.288991]
 [ 0.527247  0.262332  0.35722   0.456288  0.44198   0.43652   1.        0.672936]
 [ 0.669284  0.574497  0.635266  0.541684  0.263062  0.288991  0.672936  1.      ]]
    \end{lstlisting}
    Compare: there is slight difference, between sample covariance matricies of weekly log return and percentage return, between sample correlation matricies of weekly log return and percentage return; same holds between monthly log and percentage return. But the difference between log and percentage return becomes larger from weekly to monthly.
  \end{homeworkSection}
\end{homeworkProblem}

%----------------------------------------------------------------------------------------
%   PROBLEM 14
%----------------------------------------------------------------------------------------

\begin{homeworkProblem}
  \begin{homeworkSection}{(i)}
    The cash flows and the cash flow dates of the bonds are recorded below.
    \begin{itemize}
      \item 10 months semiannual: \$1.5 in 4 months, \$101.5 in 10 months
      \item 16 months semiannual: \$2 in 4 months, \$2 in 10 months, \$102 in 16 months
      \item 22 months annual: \$6 in 10 months, \$106 in 22 months
      \item 22 months semiannual: \$2.5 in 4 months, \$2.5 in 10 months, \$2.5 in 16 months, \$102.5 in 22 months
    \end{itemize}
  \end{homeworkSection}
  \begin{homeworkSection}{(ii)}
    This linear system can be written in matrix notation as $Ax = b$, where
    \begin{equation}
      A = \left(\begin{array}{cccc} 
      1.5 & 101.5 & 0 & 0 \\
      2 & 2 & 102 & 0 \\
      0 & 6 & 0 & 106 \\
      2.5 & 2.5 & 2.5 & 102.5
      \end{array}\right),
      x = \left(\begin{array}{c} 
      d_1 \\
      d_2 \\
      d_3 \\
      d_4
      \end{array}\right),
      b = \left(\begin{array}{c}
      101.30 \\
      102.95 \\
      107.35 \\
      105.45
      \end{array}\right)
    \end{equation}
    and $d_1, d_2, d_3, d_4$ are the discount factors correponding to the four cash flow dates, i.e, 4 months, 10 months, 16 months, 22 months. 
  \end{homeworkSection}
  \begin{homeworkSection}{(iii)}
    The solution of $Ax=b$ is obtained using the LU decomposition with row pivoting linear solver, i.e.,
    \begin{equation}
      x = linear\_solve\_lu\_row\_pivoting(A,b) = \left(\begin{array}{c} 
      0.986040243 \\
      0.983457533 \\
      0.970696122 \\
      0.957068442
      \end{array}\right)
    \end{equation}
    therefore the discount factor at 4 month is $0.986040243$, at 10 month is $0.983457533$, at 16 month is $0.970696122$, at 22 month is $0.957068442$.
  \end{homeworkSection}
  \begin{homeworkSection}{(iv)}
    According to the formula,
    \begin{equation}
      disc(t_i) = \exp(-t_ir(0,t_i))
    \end{equation}
    can be used to find the corresponding continuously compounded zero rates, as follows:
    \begin{align}
      r(0,\frac{4}{12}) = -\frac{\ln(d_1)}{4/12} = 0.04217433 \\ 
      r(0,\frac{10}{12}) = -\frac{\ln(d_2)}{10/12} = 0.02001699 \\
      r(0,\frac{16}{12}) = -\frac{\ln(d_3)}{16/12} = 0.02230636 \\
      r(0,\frac{22}{12}) = -\frac{\ln(d_4)}{22/12} = 0.02393475
    \end{align}
  \end{homeworkSection}
  \begin{homeworkSection}{(v)}
    We are looking for a function $f(x)$ of the form
    \begin{equation}
      f(x) = f_i(x) = a_i+b_ix+c_ix^2+d_ix^3, \forall x_{i-1}\le x \le x_i, \forall i=1:n
    \end{equation}
    such that
    \begin{align}
      f_i(x_{i-1}) &= v_{i-1}, \forall i = 1:n \\
      f_i(x_i) &= v_i, \forall i = 1:n \\
      f_i'(x_i) &= f_{i+1}'(x_i), \forall i = 1:(n-1) \\
      f_i''(x_i) &= f_{i+1}''(x_i), \forall i = 1:(n-1)
    \end{align}
    Two more constraints is to require that $f_1''(x_0)=0$ and $f_n''(x_n)=0$, i.e.,
    \begin{align}
      2c_1 + 6d_1x_0 &= 0 \\
      2c_n + 6d_3x_n &= 0
    \end{align}
    Let $\overline x$ be the $4n \times 1$ vector of the unknowns $a_i, b_i, c_i, d_i, i=1:n$, given by
    \begin{equation}
      \overline x(4i-3)=a_i, \overline x(4i-2)=b_i, \overline x(4i-1)=c_i, \overline x(4i)=d_i; \forall i = 1:(n-1)
    \end{equation}
    Above is a linear system with $4n$ equations and $4n$ unknowns which can be expressed in matrix notation as
    \begin{equation}
      \overline M \overline x = \overline b
    \end{equation}
    where $\overline b$ is an $4n \times 1$ vector given by
    \begin{align}
      &\overline b(1) = 0; \overline b(4n) = 0; \\
      &\overline b(4i-2) = v_{i-1}, \overline b(4i-1)=v_i, \forall i = 1:n; \\
      &\overline b(4i) = 0, \overline b(4i+1)=0, \forall i = 1:(n-1)  
    \end{align}
    and $\overline M$ is the $4n\times4n$ matrix given by (2.86) - (2.96) on textbook. \\
    We are able to solve $\overline x = \overline M^{-1} \overline b$ as of
    \begin{lstlisting}
[[ 0.04217433]
 [ 0.01248719]
 [ 0.        ]
 [-0.11238472]
 [ 0.03255237]
 [ 0.09908483]
 [-0.2597929 ]
 [ 0.14740818]
 [ 0.17105727]
 [-0.39953279]
 [ 0.33854823]
 [-0.09192827]
 [-0.09293444]
 [ 0.19444854]
 [-0.10693777]
 [ 0.01944323]]
    \end{lstlisting}
    that is 
    \begin{equation}
      f(x) = \begin{cases}
        0.04217433 + 0.01248719x -0.11238472x^3, &\text{ if }0 \le x \le 4/12 \\
        0.03255237 + 0.09908483x -0.2597929x^2 + 0.14740818x^3, &\text{ if }4/12 \le x \le 10/12\\
        0.17105727 - 0.39953279x + 0.33854823x^2 - 0.09192827x^3, &\text{ if }10/12 \le x \le 16/12\\
        -0.09293444 + 0.19444854x -0.10693777x^2 + 0.01944323x^3, &\text{ if }16/12 \le x \le 22/12
      \end{cases}
    \end{equation}
  \end{homeworkSection}
  \begin{homeworkSection}{(v)}
    To derive the value of 20 months quarterly bond with 3\% coupon rate, we need zero rate
    \begin{align}
      r(0,t_i) = f(t_i)
    \end{align}
    for $t_{i=1:7}=2/12,5/12,\dots,20/12$. Then calculate the discount factors as
    \begin{equation}
      disc(t_i) = \exp(-t_ir(0,t_i))
    \end{equation}
    therefore the value of the bond is
    \begin{equation}
      \sum_i 0.75disc(t_i) + 100\exp(-20/12r(0,20/12)) = 101.196568
    \end{equation}
  \end{homeworkSection}
\end{homeworkProblem}

%----------------------------------------------------------------------------------------
%   PROBLEM 15
%----------------------------------------------------------------------------------------

\begin{homeworkProblem}
  \begin{homeworkSection}{(i)}
    According to the formula,
    \begin{equation}
      disc(t_i) = \exp(-t_ir(0,t_i))
    \end{equation}
    can be used to find the corresponding continuously compounded zero rates, as follows:
    \begin{align}
      r(0,\frac{2}{12}) &= -\frac{\ln(d_1)}{2/12} = 0.012012016 \\ 
      r(0,\frac{5}{12}) &= -\frac{\ln(d_2)}{5/12} = 0.015650921 \\
      r(0,\frac{11}{12}) &= -\frac{\ln(d_3)}{11/12} = 0.019815241 \\
      r(0,\frac{15}{12}) &= -\frac{\ln(d_4)}{15/12} = 0.018205590
    \end{align}
  \end{homeworkSection}
  \begin{homeworkSection}{(ii)}
    Similar to part(iv) in last problem, we are looking for a function $f(x)$ of the form
    \begin{equation}
      f(x) = f_i(x) = a_i+b_ix+c_ix^2+d_ix^3, \forall x_{i-1}\le x \le x_i, \forall i=1:n
    \end{equation}
    such that
    \begin{align}
      f_i(x_{i-1}) &= v_{i-1}, \forall i = 1:n \\
      f_i(x_i) &= v_i, \forall i = 1:n \\
      f_i'(x_i) &= f_{i+1}'(x_i), \forall i = 1:(n-1) \\
      f_i''(x_i) &= f_{i+1}''(x_i), \forall i = 1:(n-1)
    \end{align}
    Two more constraints is to require that $f_1''(x_0)=0$ and $f_n''(x_n)=0$, i.e.,
    \begin{align}
      2c_1 + 6d_1x_0 &= 0 \\
      2c_n + 6d_3x_n &= 0
    \end{align}
    Let $\overline x$ be the $4n \times 1$ vector of the unknowns $a_i, b_i, c_i, d_i, i=1:n$, given by
    \begin{equation}
      \overline x(4i-3)=a_i, \overline x(4i-2)=b_i, \overline x(4i-1)=c_i, \overline x(4i)=d_i; \forall i = 1:(n-1)
    \end{equation}
    Above is a linear system with $4n$ equations and $4n$ unknowns which can be expressed in matrix notation as
    \begin{equation}
      \overline M \overline x = \overline b
    \end{equation}
    where $\overline b$ is an $4n \times 1$ vector given by
    \begin{align}
      &\overline b(1) = 0; \overline b(4n) = 0; \\
      &\overline b(4i-2) = v_{i-1}, \overline b(4i-1)=v_i, \forall i = 1:n; \\
      &\overline b(4i) = 0, \overline b(4i+1)=0, \forall i = 1:(n-1)  
    \end{align}
    and $\overline M$ is the $4n\times4n$ matrix given by (2.86) - (2.96) on textbook. \\
    In this problem, $\overline b$ is assigned as
    \begin{lstlisting}
[[ 0.      ]
 [ 0.0075  ]
 [ 0.012012]
 [ 0.      ]
 [ 0.      ]
 [ 0.012012]
 [ 0.015651]
 [ 0.      ]
 [ 0.      ]
 [ 0.015651]
 [ 0.019815]
 [ 0.      ]
 [ 0.      ]
 [ 0.019815]
 [ 0.018206]
 [ 0.      ]]
    \end{lstlisting}
    and $\overline M$ is assigned as
    \begin{lstlisting}
    0         1         2         3   4         5         6         7   8   \
0    0  0.000000  2.000000  0.000000   0  0.000000  0.000000  0.000000   0   
1    1  0.000000  0.000000  0.000000   0  0.000000  0.000000  0.000000   0   
2    1  0.166667  0.027778  0.004630   0  0.000000  0.000000  0.000000   0   
3    0  1.000000  0.333333  0.083333   0 -1.000000 -0.333333 -0.083333   0   
4    0  0.000000  2.000000  1.000000   0  0.000000 -2.000000 -1.000000   0   
5    0  0.000000  0.000000  0.000000   1  0.166667  0.027778  0.004630   0   
6    0  0.000000  0.000000  0.000000   1  0.416667  0.173611  0.072338   0   
7    0  0.000000  0.000000  0.000000   0  1.000000  0.833333  0.520833   0   
8    0  0.000000  0.000000  0.000000   0  0.000000  2.000000  2.500000   0   
9    0  0.000000  0.000000  0.000000   0  0.000000  0.000000  0.000000   1   
10   0  0.000000  0.000000  0.000000   0  0.000000  0.000000  0.000000   1   
11   0  0.000000  0.000000  0.000000   0  0.000000  0.000000  0.000000   0   
12   0  0.000000  0.000000  0.000000   0  0.000000  0.000000  0.000000   0   
13   0  0.000000  0.000000  0.000000   0  0.000000  0.000000  0.000000   0   
14   0  0.000000  0.000000  0.000000   0  0.000000  0.000000  0.000000   0   
15   0  0.000000  0.000000  0.000000   0  0.000000  0.000000  0.000000   0   

          9         10        11  12        13        14        15  
0   0.000000  0.000000  0.000000   0  0.000000  0.000000  0.000000  
1   0.000000  0.000000  0.000000   0  0.000000  0.000000  0.000000  
2   0.000000  0.000000  0.000000   0  0.000000  0.000000  0.000000  
3   0.000000  0.000000  0.000000   0  0.000000  0.000000  0.000000  
4   0.000000  0.000000  0.000000   0  0.000000  0.000000  0.000000  
5   0.000000  0.000000  0.000000   0  0.000000  0.000000  0.000000  
6   0.000000  0.000000  0.000000   0  0.000000  0.000000  0.000000  
7  -1.000000 -0.833333 -0.520833   0  0.000000  0.000000  0.000000  
8   0.000000 -2.000000 -2.500000   0  0.000000  0.000000  0.000000  
9   0.416667  0.173611  0.072338   0  0.000000  0.000000  0.000000  
10  0.916667  0.840278  0.770255   0  0.000000  0.000000  0.000000  
11  1.000000  1.833333  2.520833   0 -1.000000 -1.833333 -2.520833  
12  0.000000  2.000000  5.500000   0  0.000000 -2.000000 -5.500000  
13  0.000000  0.000000  0.000000   1  0.916667  0.840278  0.770255  
14  0.000000  0.000000  0.000000   1  1.250000  1.562500  1.953125  
15  0.000000  0.000000  0.000000   0  0.000000  2.000000  7.500000  
    \end{lstlisting}
  \end{homeworkSection}
  \begin{homeworkSection}{(iii)}
    We are able to solve $\overline x = \overline M^{-1} \overline b$ as of
    \begin{lstlisting}
[[ 0.0075  ]
 [ 0.029633]
 [ 0.      ]
 [-0.092201]
 [ 0.006767]
 [ 0.042825]
 [-0.079149]
 [ 0.066096]
 [ 0.012908]
 [-0.001391]
 [ 0.026968]
 [-0.018797]
 [-0.020615]
 [ 0.108322]
 [-0.092719]
 [ 0.024725]]
    \end{lstlisting}
    that is
    \begin{equation}
      f(x) = \begin{cases}
        0.0075 + 0.029633x - 0.092201x^3, &\text{ if }0 \le x \le 2/12\\
        0.006767 + 0.042825x -0.079149x^2 + 0.066096x^3, &\text{ if }2/12 \le x \le 5/12\\
        0.012908 -0.001391x + 0.026968x^2 -0.018797x^3, &\text{ if }5/12 \le x \le 11/12\\
        -0.020615 + 0.108322x -0.092719x^2 + 0.024725x^3, &\text{ if }11/12 \le x \le 15/12
      \end{cases}
    \end{equation}
  \end{homeworkSection}
  \begin{homeworkSection}{(iv)}
    For the value of a 14 months quarterly coupon bond with 2.5\% coupon rate, the cash flow dates are
    \begin{equation}
      t_i = 2/12, 5/12, 8/12, 11/12, 14/12
    \end{equation}
    We've known the discount factor at 2,5,11 months, so we only need to see the cash flow rate 
    \begin{align}
      disc(8/12) = exp(-8/12r(0,8/12)) &= 0.987810064\\
      disc(14/12) = exp(-14/12r(0,14/12)) &= 0.977147387
    \end{align}
    therefore the value of the bond is
    \begin{equation}
      \sum_i 0.625disc(t_i) + 100disc(14/12) = 100.915206
    \end{equation}
  \end{homeworkSection}
\end{homeworkProblem}

%----------------------------------------------------------------------------------------
%   PROBLEM 16
%----------------------------------------------------------------------------------------

\begin{homeworkProblem}
    If the $N$ day $C\%$ VaR of the portfolio is denoted by $VaR(N,C)$, where $N$ is the number of days in the time horizon and $C$ is the confidence level, then
    \begin{equation}
      VaR(N,C) \approx \sigma_V z_C \sqrt{\frac{N}{252}}V(0)
    \end{equation}
    From the formula above, it follows that
    \begin{align}
      VaR(2, 95\%) &\approx \frac{z_{95}\sqrt{2}}{z_{99}\sqrt{2}} VaR(2, 99\%) = 7.070540\\
      VaR(5, 95\%) &\approx \frac{z_{95}\sqrt{5}}{z_{99}\sqrt{2}} VaR(2, 99\%) = 11.179506
    \end{align}
    where the unit are million dollars.
\end{homeworkProblem}

%----------------------------------------------------------------------------------------
%   PROBLEM 17
%----------------------------------------------------------------------------------------

\begin{homeworkProblem}
  \begin{homeworkSection}{(i)}
     The asset allocation for the tangenry portfolio is
     \begin{equation}
       w_T = \frac{1}{\textbf{1}^t\Sigma_R^{-1}\overline\mu}\Sigma_R^{-1}\overline\mu = [0.173699,0.338145,0.173604,0.314552]^t
     \end{equation}
     The weight of the cash position of the portfolio is
     \begin{equation}
       w_{cash} = 1 - \textbf{1}^tw_T = 0
     \end{equation}
     The expected value of the return is
     \begin{equation}
       \mu_T = r_f + \overline \mu^tw_T = 0.038158
     \end{equation}
     The standard deviation of the return is
     \begin{equation}
       \sigma_T = \sqrt{w_T^t\Sigma_Rw_T} = 0.135302
     \end{equation}
     The Sharpe ratio is
     \begin{equation}
       \frac{\mu_T-r_f}{\sigma_T} = 0.208112
     \end{equation}
  \end{homeworkSection}
  \begin{homeworkSection}{(ii)}
    The asset allocation for a minimum variance portfolio with 3\% expected return is
    \begin{equation}
      w_{min} = \frac{\mu_P - r_f}{\overline\mu^t\Sigma_R^{-1}\overline\mu} \Sigma_R^{-1} \overline\mu = [0.123374,0.240177,0.123307,0.223419]^t
    \end{equation}
    The weight of the cash position is
    \begin{equation}
      w_{min,cash} = 1 - \textbf{1}^tw_{min} = 0.289722
    \end{equation}
    The standard deviation of the return is
    \begin{equation}
      \sigma_{min} = \sqrt{w_{min}^t\Sigma_Rw_{min}} = 0.096102
    \end{equation}
    The Sharpe ratio is
    \begin{equation}
      \frac{\mu_P-r_f}{\sigma_{min}} = 0.208112
    \end{equation}
  \end{homeworkSection}
  \begin{homeworkSection}{(iii)}
    The asset allocation for a maximum return portfolio with 27\% standard deviation of return is
    \begin{equation}
      w_{max} = \frac{\sigma_P}{\sqrt{\overline\mu\Sigma_R^{-1}\overline\mu}}\Sigma_R^{-1}\overline\mu = [0.346622,0.674782,0.346433,0.627700]^t
    \end{equation}
    The weight of the cash position is
    \begin{equation}
      w_{max,cash} = 1 - \textbf{1}^tw_{max} = -0.995537
    \end{equation}
    The expected value of the return is
    \begin{equation}
      \mu_{max} = r_f + \overline \mu^tw = 0.066190
    \end{equation}
    The Sharpe ratio is
    \begin{equation}
      \frac{\mu_{max}-r_f}{\sigma_P} = 0.208112
    \end{equation}
  \end{homeworkSection}
  \begin{homeworkSection}{(iv)}
    The asset allocation of min variance without cash weight is 
    \begin{equation}
      w_{m} = \frac{\sigma_P}{\sqrt{\overline\mu\Sigma_R^{-1}\overline\mu}}\Sigma_R^{-1}\overline\mu = [0.190503,0.339011,0.089277,0.381209]^t
    \end{equation}
    The expected value of the return is
    \begin{equation}
      \mu_m = r_f + \overline \mu^tw = 0.036910
    \end{equation}
    The standard deviation of the return is
    \begin{equation}
      \sigma_m = \sqrt{w_{m}^t\Sigma_Rw_{m}} = 0.132271
    \end{equation}
    The Sharpe ratio is
    \begin{equation}
      \frac{\mu_m-r_f}{\sigma_m} = 0.203450
    \end{equation}
  \end{homeworkSection}
\end{homeworkProblem}

%----------------------------------------------------------------------------------------
%   PROBLEM 18
%----------------------------------------------------------------------------------------

\begin{homeworkProblem}
  \begin{homeworkSection}{(i)}
    Given the correlation matrix as 
    \begin{equation}
      \Omega_X = \left(\begin{array} {ccc}
      1 & -0.25 & 0.5 \\
      -0.25 & 1 & 0.25 \\
      0.5 & 0.25 & 1
      \end{array} \right)
    \end{equation}
    and the standard deviations, we are able to calculate the covariance matrix
    \begin{equation}
      \Sigma_X = D\Omega_XD = \left(\begin{array} {ccc}
      0.0225 & -0.0075 & 0.01875 \\
      -0.0075 & 0.04 & 0.0125 \\
      0.01875 & 0.0125 & 0.0625
      \end{array} \right)
    \end{equation}
    Similar to last problem to find the minimum variance portfolio with 8\% expected return:
    \begin{align}
      w_{min} &= \frac{\mu_P - r_f}{\overline\mu^t\Sigma_R^{-1}\overline\mu} \Sigma_R^{-1} \overline\mu = [0.455385,0.535385,0.110769]^t \\
      w_{min,cash} &= 1 - \textbf{1}^tw_{max} =  -0.101538 \\
      \sigma_{min} &= \sqrt{w_{min}^t\Sigma_Xw_{min}} = 0.128901
    \end{align}
    where $w_{min}$ is the asset allocation, $w_{min,cash}$ is the cash allocation, and $\sigma_{min}$ is the standard deviation of the return of this portfolio.
  \end{homeworkSection}
  \begin{homeworkSection}{(ii)}
    According to the description, we have the returns of the minimum variance portfolio as $Y \sim N(\mu_Y,\Sigma_Y)$, where
    \begin{equation}
      \mu_Y = r_f + w^t\overline \mu = 0.08
    \end{equation}
    and with $Y = r_f + w^tX$, then
    \begin{equation}
      \Sigma_Y = w^t\Sigma_Xw = 0.01661538
    \end{equation}
    The probability density function $f$ of a nondegenerate multivariate normal variable $Y$ is given by
    \begin{align}
      f(x) &= \frac{1}{(2\pi)^{1/2}\sqrt{\Sigma_Y}} \exp(-\frac{(x-\mu_Y)^2}{2\times\Sigma_Y}) \\
      &= \frac{1}{(2\pi)^{1/2}0.128901} \exp(-\frac{(x-0.08)^2}{2\times0.01661538})
    \end{align}
    where the values of $\Sigma_Y$ and $\mu_Y$ are given before.
  \end{homeworkSection}
  \begin{homeworkSection}{(iii)}
    According to last part,
    \begin{equation}
      Y \sim N(\mu_Y, \Sigma_Y)
    \end{equation}
    Therefore, the probability that the return of the minimum variance portfolio with 8\% expected return is between 7\% and 9\% is
    \begin{align}
      Pr(0.07 < Y < 0.09) &= Pr(\frac{0.07-\mu_Y}{\sqrt{\Sigma_Y}} < Z < \frac{0.09-\mu_Y}{\sqrt{\Sigma_Y}}) \\
      &= N(\frac{0.09-\mu_Y}{\sqrt{\Sigma_Y}}) - N(\frac{0.07-\mu_Y}{\sqrt{\Sigma_Y}}) \\
      &= 0.061837
    \end{align}
    where $Z$ is the random variable of standard normal distribution and $N()$ is the cdf of standard normal distribution. Similarly,
    \begin{align}
      Pr(Y < 0.05) &= N(\frac{0.05-\mu_Y}{\sqrt{\Sigma_Y}}) = 0.407983 \\
      Pr(Y > 0.1) &= 1 - N(\frac{0.1-\mu_Y}{\sqrt{\Sigma_Y}}) = 0.438348
    \end{align} 
  \end{homeworkSection}
  \begin{homeworkSection}{(iv)}
    If the portfolio equally invested in each of the three assets, then $w' = (1/3,1/3,1/3)^t$, similarly
    \begin{align}
      \mu_Y' &= 0.08 \\
      \Sigma_Y' &= w'^t\Sigma_Xw' = 0.0191666666667 
    \end{align}
    Then $Y' \sim N(\mu_Y', \Sigma_Y')$, the probabilities questioned are
    \begin{align}
      Pr(0.07 < Y' < 0.09) &= N(\frac{0.09-\mu_Y'}{\sqrt{\Sigma_Y'}}) - N(\frac{0.07-\mu_Y'}{\sqrt{\Sigma_Y'}}) = 0.057582 \\
      Pr(Y'<0.05) &= N(\frac{0.05-\mu_Y'}{\sqrt{\Sigma_Y'}}) = 0.414223 \\
      Pr(Y'>0.1) &= 1 - N(\frac{0.1-\mu_Y'}{\sqrt{\Sigma_Y'}}) = 0.442567
    \end{align}
  \end{homeworkSection}
\end{homeworkProblem}

%----------------------------------------------------------------------------------------
%   PROBLEM 19
%----------------------------------------------------------------------------------------

\begin{homeworkProblem}
  \begin{homeworkSection}{(i)}
    From the approximate VaR formula,
    \begin{equation}
      VaR(N,C) \approx (\sqrt{\frac{N}{252}} \sigma_{R_V}z_C - \frac{N}{252}\mu_{R_V}) V(0) 
    \end{equation}
    we obtain the 5-day 95\% VaR of \$100 million portfolios fully invested in one asset
    \begin{align}
      VaR(5,95\%) &\approx (\sqrt{\frac{5}{252}} \sigma_1 z_{95} - \frac{5}{252}\mu_1) \cdot \$100mil = \$5.633582mil\\
      VaR(5,95\%) &\approx (\sqrt{\frac{5}{252}} \sigma_2 z_{95} - \frac{5}{252}\mu_2) \cdot \$100mil = \$5.554217mil\\
      VaR(5,95\%) &\approx (\sqrt{\frac{5}{252}} \sigma_3 z_{95} - \frac{5}{252}\mu_3) \cdot \$100mil = \$6.633314mil
    \end{align}
  \end{homeworkSection}
  \begin{homeworkSection}{(ii)}
    To minimize the VaR given $N,C,V(0)$ is to minimize the standard deviation, i.e., we will invest in the minimum variance portfolio fully invested in two of the three assets in this part. Recall the standard deviation of the return of the minimum variance portfolio fully invested is
    \begin{equation}
      \sigma_m = \frac{1}{\sqrt{\textbf{1}^t\Sigma_R^{-1}\textbf{1}}}
    \end{equation}
    When investing in the first and second asset, $\sigma_{m1}=0.153093109$, then the VaR is
    \begin{equation}
      VaR_1(5,95\%) \approx (\sqrt{\frac{5}{252}} \sigma_{m1} z_{95} - \frac{5}{252}\mu_{m1}) \cdot \$100mil = \$3.348640mil
    \end{equation}
    When investing in the second and third asset, $\sigma_{m2}=0.166598122$, then the VaR is
    \begin{equation}
      VaR_2(5,95\%) \approx (\sqrt{\frac{5}{252}} \sigma_{m2} z_{95} - \frac{5}{252}\mu_{m2}) \cdot \$100mil = \$3.587919mil
    \end{equation}
    When investing in the first and third asset, $\sigma_{m3}=0.214140090$, then the VaR is
    \begin{equation}
      VaR_3(5,95\%) \approx (\sqrt{\frac{5}{252}} \sigma_{m3} z_{95} - \frac{5}{252}\mu_{m3}) \cdot \$100mil = \$4.742349mil
    \end{equation}
  \end{homeworkSection}
  \begin{homeworkSection}{(iii)}
    Similar to part(ii), when investing in all three assets,
    \begin{equation}
       \sigma_m = \frac{1}{\sqrt{\textbf{1}^t\Sigma_R^{-1}\textbf{1}}} = 0.134824841
    \end{equation}
    and the VaR is
    \begin{equation}
      VaR(5,95\%) \approx (\sqrt{\frac{5}{252}} \sigma_m z_{95} - \frac{5}{252}\mu_{m}) \cdot \$100mil = \$2.894387mil
    \end{equation}
  \end{homeworkSection}
\end{homeworkProblem}

%----------------------------------------------------------------------------------------
%   PROBLEM 20
%----------------------------------------------------------------------------------------

\begin{homeworkProblem}
  \begin{homeworkSection}{(i)}
    The problem is same as solving 
    \begin{equation}
      \overline C - \overline P = PVF - K\cdot disc 
    \end{equation}
    It follows that the values of $PVF$ and $disc$ can be obtained by solving a least squares problem $y \approx Ax$ with $x = (PVF, disc)^t$ and with $18\times1$ matrix $A$ and the $16\times1$ column vector $y$ corresponding to $\overline C - \overline P$ for each strike. \\
    The solution is $x=(A^tA)^{-1}A^ty$ to this least squares problem, computed as $x=least\_squares(A,y)$, we obtain
    \begin{equation}
      x = \left(\begin{array}{c}
      PVF \\
      disc
      \end{array} \right) = \left(\begin{array}{c}
      1869.403121 \\
      0.997227746
      \end{array} \right)
    \end{equation}
  \end{homeworkSection}
  \begin{homeworkSection}{(ii)}
    This is using Newton's method recursion for solving
    \begin{align}
      f_C(x) &= PVF \cdot N(d_1(x)) - K \cdot N(d_2(x)) - C_m  \\
      f_P(x) &= -PVF \cdot N(-d_1(x)) + K \cdot N(-d_2(x)) - P_m 
    \end{align}
    with $d_1(x)$ and $d_2(x)$ given as
    \begin{equation}
      d_1(x) = \frac{\ln(PVF/(K\cdot disc))}{x\sqrt{T}} + \frac{x\sqrt{T}}{2}, d_2(x) = \frac{\ln(PVF/(K\cdot disc))}{x\sqrt{T}} - \frac{x\sqrt{T}}{2}
    \end{equation}
    The newton's method recursion is 
    \begin{align}
      x_{k+1} &= x_k - \frac{f_C(x_k)}{f'_C(x_k)} \\
      x_{k+1} &= x_k - \frac{f_C(x_k)}{f'_P(x_k)}
    \end{align}
    After calculations, the implied volatilities are as follows (note that the $T$ above is set as $240/365$):
    \begin{lstlisting}
Strike  Imp_Vol(call)   Imp_Vol(put)
1450    0.217612        0.218549
1500    0.207823        0.208539
1550    0.198264        0.198997
1600    0.189059        0.189635
1675    0.175484        0.175236
1700    0.170474        0.170174
1750    0.160755        0.160567
1775    0.15649         0.155881
1800    0.151343        0.151331
1825    0.146385        0.146148
1850    0.141347        0.141149
1875    0.136558        0.136063
1900    0.13159         0.131458
1925    0.126686        0.126247
1975    0.117725        0.11731
2000    0.113451        0.114061
2050    0.106157        0.1072
2100    0.100717        0.102236
    \end{lstlisting}
    The implied volatilities from the output corresponding to calls and puts with the same strike are nearly identical, but if more precisely, the implied volatilities corresponding to puts are always a bit larger than the implied volatilities corresponding to calls.
  \end{homeworkSection}
\end{homeworkProblem}

%----------------------------------------------------------------------------------------
%   PROBLEM 21
%----------------------------------------------------------------------------------------

\begin{homeworkProblem}
  \begin{homeworkSection}{(i)}
    This can be written in least squares form as $y \approx Ax$, with
    \begin{equation}
      x = \left( \begin{array}{c} 
      a \\
      b_1 \\
      b_2 \\
      b_3
      \end{array} \right); y = T_3; A = (1,T_2,T_5,T_{10})
    \end{equation}
    Recall from the textbook the solution to the least squares problem is
    \begin{equation}
      x \approx (A^tA)^{-1}A^ty
    \end{equation}
    By using the Cholesky solver, we compute $x=linear\_solver\_choleasky(A^tA,A^ty)$ with $A$ and $y$ given above, we obtain that
    \begin{equation}
      x = \left( \begin{array}{c} 
      a \\
      b_1 \\
      b_2 \\
      b_3
      \end{array} \right) = \left( \begin{array}{c} 
      0.012302 \\
      0.127208 \\
      0.334045 \\
      0.529777
      \end{array} \right)
    \end{equation}
    We conclude that the ordinary least square linear regression for the yield of the 3-year bond in terms of the yields of the 2-year, 5-year, and 10-year bonds is
    \begin{equation}
      T_3 \approx 0.012302 \cdot \textbf{1} + 0.127208 T_2 + 0.334045 T_5 + 0.529777T_{10}
    \end{equation}
    The approximation error is
    \begin{equation}
      \text{error}_{linear\_interp} = ||T_3 - T_{3,LR}|| = 0.043013
    \end{equation}
  \end{homeworkSection}
  \begin{homeworkSection}{(ii)}
    Given the linear interpolation formula
    \begin{equation}
      T_3 \approx T_{3,linear\_interp} = \frac{2}{3}T_2 + \frac{1}{3}T_5 
    \end{equation}
    The approximation error is
    \begin{equation}
      \text{error}_{linear\_interp} = ||T_3 - T_{3,linear\_interp}|| = 0.206613
    \end{equation}
  \end{homeworkSection}
  \begin{homeworkSection}{(iii)}
    Similar to problem 13 and problem 14, we need to apply cubic spline interpolation. We are looking for a function $f(x)$ of the form
    \begin{equation}
      f(x) = f_i(x) = a_i+b_ix+c_ix^2+d_ix^3, \forall x_{i-1}\le x \le x_i, \forall i=1:n
    \end{equation}
    such that
    \begin{align}
      f_i(x_{i-1}) &= v_{i-1}, \forall i = 1:n \\
      f_i(x_i) &= v_i, \forall i = 1:n \\
      f_i'(x_i) &= f_{i+1}'(x_i), \forall i = 1:(n-1) \\
      f_i''(x_i) &= f_{i+1}''(x_i), \forall i = 1:(n-1)
    \end{align}
    Two more constraints is to require that $f_1''(x_0)=0$ and $f_n''(x_n)=0$, i.e.,
    \begin{align}
      2c_1 + 6d_1x_0 &= 0 \\
      2c_n + 6d_3x_n &= 0
    \end{align}
    Let $\overline x$ be the $4n \times 1$ vector of the unknowns $a_i, b_i, c_i, d_i, i=1:n$, given by
    \begin{equation}
      \overline x(4i-3)=a_i, \overline x(4i-2)=b_i, \overline x(4i-1)=c_i, \overline x(4i)=d_i; \forall i = 1:(n-1)
    \end{equation}
    Above is a linear system with $4n$ equations and $4n$ unknowns which can be expressed in matrix notation as
    \begin{equation}
      \overline M \overline x = \overline b
    \end{equation}
    where $\overline b$ is an $4n \times 1$ vector given by
    \begin{align}
      &\overline b(1) = 0; \overline b(4n) = 0; \\
      &\overline b(4i-2) = v_{i-1}, \overline b(4i-1)=v_i, \forall i = 1:n; \\
      &\overline b(4i) = 0, \overline b(4i+1)=0, \forall i = 1:(n-1)  
    \end{align}
    and $\overline M$ is the $4n\times4n$ matrix given by (2.86) - (2.96) on textbook. \\
    In this problem,
    \begin{equation}
      x_{0:2} = 2,5,10; v_{0:2} = T_2[i], T_5[i], T_{10}[i]; n =2
    \end{equation}
    for $i=1:15$. \\
    We are able to solve $\overline x = \overline M^{-1} \overline b$. As for the first example, $\overline x_1$ is
    \begin{lstlisting}
[[  4.85328700e+00]
 [ -2.18580000e-02]
 [ -3.83900000e-03]
 [  6.40000000e-04]
 [  4.98125000e+00]
 [ -9.86360000e-02]
 [  1.15170000e-02]
 [ -3.84000000e-04]]
    \end{lstlisting}
    that is 
    \begin{equation}
      f(x) = \begin{cases}
        4.853287 -0.021858x -0.003839x^2 +0.000640x^3, &\text{ if }2 \le x \le 5 \\
        4.981250 -0.098636x +0.011517x^2 -0.000384x^3, &\text{ if }5 \le x \le 10
      \end{cases}
    \end{equation}
    Therefore after doing the same stuffs the other 14 datasets, the approximation error is
    \begin{equation}
      \text{error}_{cubic\_interp} = ||T_3 - T_{3,cubic\_interp}|| = 0.186435
    \end{equation}
  \end{homeworkSection}
  \begin{homeworkSection}{(iv)}
    Compare: the ordinary least squares method gives the best answer with lowest error, better than the other two methods using interpolation. For the other two, cubic interpolation generates lower error than linear interpolation, which is expected since using more parameters.
  \end{homeworkSection}
\end{homeworkProblem}

\end{document}
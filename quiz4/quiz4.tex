%%%%%%%%%%%%%%%%%%%%%%%%%%%%%%%%%%%%%%%%%
% Structured General Purpose Assignment
% LaTeX Template
%
% This template has been downloaded from:
% http://www.latextemplates.com
%
% Original author:
% Ted Pavlic (http://www.tedpavlic.com)
%
% Note:
% The \lipsum[#] commands throughout this template generate dummy text
% to fill the template out. These commands should all be removed when 
% writing assignment content.
%
%%%%%%%%%%%%%%%%%%%%%%%%%%%%%%%%%%%%%%%%%

%----------------------------------------------------------------------------------------
%	PACKAGES AND OTHER DOCUMENT CONFIGURATIONS
%----------------------------------------------------------------------------------------

\documentclass{article}

\usepackage{fancyhdr} % Required for custom headers
\usepackage{lastpage} % Required to determine the last page for the footer
\usepackage{extramarks} % Required for headers and footers
\usepackage{graphicx} % Required to insert images
\usepackage{lipsum} % Used for inserting dummy 'Lorem ipsum' text into the template
\usepackage{listings}
\usepackage{color}
\usepackage{amsmath}

\definecolor{dkgreen}{rgb}{0,0.6,0}
\definecolor{gray}{rgb}{0.5,0.5,0.5}
\definecolor{mauve}{rgb}{0.58,0,0.82}

\lstset{frame=tb,
  language=Python,
  aboveskip=3mm,
  belowskip=3mm,
  showstringspaces=false,
  columns=flexible,
  basicstyle={\small\ttfamily},
  numbers=none,
  numberstyle=\tiny\color{gray},
  keywordstyle=\color{blue},
  commentstyle=\color{dkgreen},
  stringstyle=\color{mauve},
  breaklines=true,
  breakatwhitespace=true
  tabsize=3
}

% Margins
\topmargin=-0.45in
\evensidemargin=0in
\oddsidemargin=0in
\textwidth=6.5in
\textheight=9.0in
\headsep=0.25in 

\linespread{1.1} % Line spacing

% Set up the header and footer
\pagestyle{fancy}
\lhead{\hmwkAuthorName} % Top left header
\chead{\hmwkClass\ (\hmwkClassInstructor\ \hmwkClassTime): \hmwkTitle} % Top center header
\rhead{\firstxmark} % Top right header
\lfoot{\lastxmark} % Bottom left footer
\cfoot{} % Bottom center footer
\rfoot{Page\ \thepage\ of\ \pageref{LastPage}} % Bottom right footer
\renewcommand\headrulewidth{0.4pt} % Size of the header rule
\renewcommand\footrulewidth{0.4pt} % Size of the footer rule

\setlength\parindent{0pt} % Removes all indentation from paragraphs

%----------------------------------------------------------------------------------------
%	DOCUMENT STRUCTURE COMMANDS
%	Skip this unless you know what you're doing
%----------------------------------------------------------------------------------------

% Header and footer for when a page split occurs within a problem environment
\newcommand{\enterProblemHeader}[1]{
\nobreak\extramarks{#1}{#1 continued on next page\ldots}\nobreak
\nobreak\extramarks{#1 (continued)}{#1 continued on next page\ldots}\nobreak
}

% Header and footer for when a page split occurs between problem environments
\newcommand{\exitProblemHeader}[1]{
\nobreak\extramarks{#1 (continued)}{#1 continued on next page\ldots}\nobreak
\nobreak\extramarks{#1}{}\nobreak
}

\setcounter{secnumdepth}{0} % Removes default section numbers
\newcounter{homeworkProblemCounter} % Creates a counter to keep track of the number of problems

\newcommand{\homeworkProblemName}{}
\newenvironment{homeworkProblem}[1][Problem \arabic{homeworkProblemCounter}]{ % Makes a new environment called homeworkProblem which takes 1 argument (custom name) but the default is "Problem #"
\stepcounter{homeworkProblemCounter} % Increase counter for number of problems
\renewcommand{\homeworkProblemName}{#1} % Assign \homeworkProblemName the name of the problem
\section{\homeworkProblemName} % Make a section in the document with the custom problem count
\enterProblemHeader{\homeworkProblemName} % Header and footer within the environment
}{
\exitProblemHeader{\homeworkProblemName} % Header and footer after the environment
}

\newcommand{\problemAnswer}[1]{ % Defines the problem answer command with the content as the only argument
\noindent\framebox[\columnwidth][c]{\begin{minipage}{0.98\columnwidth}#1\end{minipage}} % Makes the box around the problem answer and puts the content inside
}

\newcommand{\homeworkSectionName}{}
\newenvironment{homeworkSection}[1]{ % New environment for sections within homework problems, takes 1 argument - the name of the section
\renewcommand{\homeworkSectionName}{#1} % Assign \homeworkSectionName to the name of the section from the environment argument
\subsection{\homeworkSectionName} % Make a subsection with the custom name of the subsection
\enterProblemHeader{\homeworkProblemName\ [\homeworkSectionName]} % Header and footer within the environment
}{
\enterProblemHeader{\homeworkProblemName} % Header and footer after the environment
}
   
%----------------------------------------------------------------------------------------
%	NAME AND CLASS SECTION
%----------------------------------------------------------------------------------------

\newcommand{\hmwkTitle}{Quiz 4} % Assignment title
\newcommand{\hmwkDueDate}{Aug 21,\ 2014} % Due date
\newcommand{\hmwkClass}{Numerical Linear Algebra} % Course/class
\newcommand{\hmwkClassTime}{6:00 pm} % Class/lecture time
\newcommand{\hmwkClassInstructor}{Lecture time:} % Teacher/lecturer
\newcommand{\hmwkAuthorName}{Weiyi Chen} % Your name

%----------------------------------------------------------------------------------------
%	TITLE PAGE
%----------------------------------------------------------------------------------------

\title{
\vspace{2in}
\textmd{\textbf{\hmwkClass:\ \hmwkTitle}}\\
\normalsize\vspace{0.1in}\small{Due\ on\ \hmwkDueDate}\\
\vspace{0.1in}\large{\textit{\hmwkClassInstructor\ \hmwkClassTime}}
\vspace{3in}
}

\author{\textbf{\hmwkAuthorName}}
\date{} % Insert date here if you want it to appear below your name

%----------------------------------------------------------------------------------------

\begin{document}

\maketitle

%----------------------------------------------------------------------------------------
%	TABLE OF CONTENTS
%----------------------------------------------------------------------------------------

%\setcounter{tocdepth}{1} % Uncomment this line if you don't want subsections listed in the ToC

%\newpage
%\tableofcontents
\newpage

%----------------------------------------------------------------------------------------
%   PROBLEM 1
%----------------------------------------------------------------------------------------

\begin{homeworkProblem}
  Since the linear combination of normal distribution is still normal distribute, we let
  \begin{equation}
    X_1 = w_1^t Z = (w_{11}, w_{12})(Z_1, Z_2)^t, X_2 = w_2^t Z = (w_{21}, w_{22})(Z_1, Z_2)^t
  \end{equation}
  Then according to the covariance matrix, we have
  \begin{align}
    var(X_1) &= w_{11}^2 + w_{12}^2 = w_1^2 = 4 \\
    var(X_2) &= w_{21}^2 + w_{22}^2 = w_2^2 = 9 \\
    cov(X_1, X_2) &= w_{11}w_{21} + w_{21}w_{22} = w_1^tw_2 =  -1 
  \end{align}
  where $w_1 = (w_{11},w_{12})^t$, $w_2 = (w_{21},w_{22})^t$. Let $W = col(w_i)$ for $i=1,2$, then
  \begin{equation}
    W^tW = \left( \begin{array} {cc}
    w_{11} & w_{12} \\
    w_{21} & w_{22}
    \end{array} \right) \left( \begin{array} {cc}
    w_{11} & w_{21} \\
    w_{12} & w_{22} 
    \end{array} \right) = \left( \begin{array} {cc} 
    4 & -1 \\
    -1 & 9
    \end{array} \right)
  \end{equation}
  One solution is just apply Cholesky decomposition (which implies $w_{12}=0$), we get
  \begin{equation}
    W = \left( \begin{array} {cc}
    2 & -0.5 \\
    0 & 2.95803989
    \end{array} \right)
  \end{equation}
  Therefore $X_1 = 2Z_1$ and $X_2 = -0.5Z_1 + 2.95803989Z_2$ is a possible pair of normal variables satisfying the covariance matrix.
\end{homeworkProblem}

%----------------------------------------------------------------------------------------
%   PROBLEM 2
%----------------------------------------------------------------------------------------

\begin{homeworkProblem}
  Similarly to last problem, let
  \begin{equation}
    W = \left(\begin{array} {ccc} 
    w_{11} & w_{12} & w_{13} \\
    w_{21} & w_{22} & w_{23} \\
    w_{31} & w_{32} & w_{33}
    \end{array} \right)
  \end{equation}
  satisfying $WZ = W(Z_1, Z_2, Z_3)^t = (X_1, X_2, X_3)^t$. Then the covariance matrix and the correlation matrix are 
  \begin{equation}
    \Sigma_X = WW^t, \Omega_X = D_{\sigma_X}^{-1}\Sigma_XD_{\sigma_X}^{-1}
  \end{equation}
  where
  \begin{equation}
    D_{\sigma_X}^{-1} = diag(\Sigma_X)^{-1} = diag(|w_1|, |w_2|, |w_3|)^{-1}
  \end{equation}
  with $w_i = (w_{i1}, w_{i2}, w_{i3})^t$ and $|w_i| = (w_{i1}^2 + w_{i2}^2 + w_{i3}^2)^{1/2}$. \\
  Since $(D_{\sigma_X}^{-1})^t = D_{\sigma_X}^{-1}$, then
  \begin{equation}
    \Omega_X = D_{\sigma_X}^{-1}\Sigma_XD_{\sigma_X}^{-1} = (W^tD_{\sigma_X}^{-1})^t(W^tD_{\sigma_X}^{-1})
  \end{equation}
  where
  \begin{equation}
    W^tD_{\sigma_X}^{-1} = col(w_i)_{i=1:3}diag(|w_i|^{-1})_{i=1:3} = col(\frac{w_i}{|w_i|})_{i=1:3}
  \end{equation}
  Again we apply Cholesky decomposition to derive $W^tD_{\sigma_X}^{-1}$ since we are given $\Omega_X$, the result is
  \begin{equation}
    W^tD_{\sigma_X}^{-1} = \left(\begin{array}{ccc} 
    1 & 0.3 & 0.4 \\
    0 & 0.9539392 & 0.39834824 \\
    0 & 0 & 0.82542031
    \end{array} \right)
  \end{equation}
  One solution of $W$ is just take the values above since $D_{\sigma_X}^{-1}$ is just to normalize each columns of $W^t$. In other words,
  \begin{equation}
    W^t = 
    \left(\begin{array}{ccc}
    w_{11} & w_{21} & w_{31} \\
    w_{12} & w_{22} & w_{32} \\
    w_{13} & w_{23} & w_{33}
    \end{array} \right) =
    \left(\begin{array}{ccc}
    1 & 0.3 & 0.4 \\
    0 & 0.9539392 & 0.39834824 \\
    0 & 0 & 0.82542031
    \end{array} \right)
  \end{equation}
  Therefore
  \begin{align}
    X_1 &= Z_1 \\ 
    X_2 &= 0.3Z_1 + 0.9539392Z_2 \\
    X_3 &= 0.4Z_1 + 0.39834824Z_2 + 0.82542031Z_3
  \end{align}
  is a possible combination of normal variables satisfying the given correlation matrix.
\end{homeworkProblem}

%----------------------------------------------------------------------------------------
%   PROBLEM 3
%----------------------------------------------------------------------------------------

\begin{homeworkProblem}
  According to the distribution of the 3-dimensional multivariate normal random variable,
  \begin{equation}
    X_1+2X_2+3X_3 \sim N \left( 1-4+3, (1,2,3) \left( \begin{array} {ccc}
    1 & -1 & 0\\
    -1 & 3 & -2 \\
    0 & -2 & 3
    \end{array} \right) \left( \begin{array} {c}
    1 \\
    2 \\
    3
    \end{array} \right)
    \right)
  \end{equation}
  or
  \begin{equation} 
    X_1+2X_2+3X_3 \sim N(0,12) 
  \end{equation}
  Therefore the probability that $X_1+2X_2+3X_3$ is positive is $p = 0.5$.
\end{homeworkProblem}

\end{document}
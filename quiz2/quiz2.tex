%%%%%%%%%%%%%%%%%%%%%%%%%%%%%%%%%%%%%%%%%
% Structured General Purpose Assignment
% LaTeX Template
%
% This template has been downloaded from:
% http://www.latextemplates.com
%
% Original author:
% Ted Pavlic (http://www.tedpavlic.com)
%
% Note:
% The \lipsum[#] commands throughout this template generate dummy text
% to fill the template out. These commands should all be removed when 
% writing assignment content.
%
%%%%%%%%%%%%%%%%%%%%%%%%%%%%%%%%%%%%%%%%%

%----------------------------------------------------------------------------------------
%	PACKAGES AND OTHER DOCUMENT CONFIGURATIONS
%----------------------------------------------------------------------------------------

\documentclass{article}

\usepackage{fancyhdr} % Required for custom headers
\usepackage{lastpage} % Required to determine the last page for the footer
\usepackage{extramarks} % Required for headers and footers
\usepackage{graphicx} % Required to insert images
\usepackage{lipsum} % Used for inserting dummy 'Lorem ipsum' text into the template
\usepackage{listings}
\usepackage{color}
\usepackage{amsmath}

\definecolor{dkgreen}{rgb}{0,0.6,0}
\definecolor{gray}{rgb}{0.5,0.5,0.5}
\definecolor{mauve}{rgb}{0.58,0,0.82}

\lstset{frame=tb,
  language=Python,
  aboveskip=3mm,
  belowskip=3mm,
  showstringspaces=false,
  columns=flexible,
  basicstyle={\small\ttfamily},
  numbers=none,
  numberstyle=\tiny\color{gray},
  keywordstyle=\color{blue},
  commentstyle=\color{dkgreen},
  stringstyle=\color{mauve},
  breaklines=true,
  breakatwhitespace=true
  tabsize=3
}

% Margins
\topmargin=-0.45in
\evensidemargin=0in
\oddsidemargin=0in
\textwidth=6.5in
\textheight=9.0in
\headsep=0.25in 

\linespread{1.1} % Line spacing

% Set up the header and footer
\pagestyle{fancy}
\lhead{\hmwkAuthorName} % Top left header
\chead{\hmwkClass\ (\hmwkClassInstructor\ \hmwkClassTime): \hmwkTitle} % Top center header
\rhead{\firstxmark} % Top right header
\lfoot{\lastxmark} % Bottom left footer
\cfoot{} % Bottom center footer
\rfoot{Page\ \thepage\ of\ \pageref{LastPage}} % Bottom right footer
\renewcommand\headrulewidth{0.4pt} % Size of the header rule
\renewcommand\footrulewidth{0.4pt} % Size of the footer rule

\setlength\parindent{0pt} % Removes all indentation from paragraphs

%----------------------------------------------------------------------------------------
%	DOCUMENT STRUCTURE COMMANDS
%	Skip this unless you know what you're doing
%----------------------------------------------------------------------------------------

% Header and footer for when a page split occurs within a problem environment
\newcommand{\enterProblemHeader}[1]{
\nobreak\extramarks{#1}{#1 continued on next page\ldots}\nobreak
\nobreak\extramarks{#1 (continued)}{#1 continued on next page\ldots}\nobreak
}

% Header and footer for when a page split occurs between problem environments
\newcommand{\exitProblemHeader}[1]{
\nobreak\extramarks{#1 (continued)}{#1 continued on next page\ldots}\nobreak
\nobreak\extramarks{#1}{}\nobreak
}

\setcounter{secnumdepth}{0} % Removes default section numbers
\newcounter{homeworkProblemCounter} % Creates a counter to keep track of the number of problems

\newcommand{\homeworkProblemName}{}
\newenvironment{homeworkProblem}[1][Problem \arabic{homeworkProblemCounter}]{ % Makes a new environment called homeworkProblem which takes 1 argument (custom name) but the default is "Problem #"
\stepcounter{homeworkProblemCounter} % Increase counter for number of problems
\renewcommand{\homeworkProblemName}{#1} % Assign \homeworkProblemName the name of the problem
\section{\homeworkProblemName} % Make a section in the document with the custom problem count
\enterProblemHeader{\homeworkProblemName} % Header and footer within the environment
}{
\exitProblemHeader{\homeworkProblemName} % Header and footer after the environment
}

\newcommand{\problemAnswer}[1]{ % Defines the problem answer command with the content as the only argument
\noindent\framebox[\columnwidth][c]{\begin{minipage}{0.98\columnwidth}#1\end{minipage}} % Makes the box around the problem answer and puts the content inside
}

\newcommand{\homeworkSectionName}{}
\newenvironment{homeworkSection}[1]{ % New environment for sections within homework problems, takes 1 argument - the name of the section
\renewcommand{\homeworkSectionName}{#1} % Assign \homeworkSectionName to the name of the section from the environment argument
\subsection{\homeworkSectionName} % Make a subsection with the custom name of the subsection
\enterProblemHeader{\homeworkProblemName\ [\homeworkSectionName]} % Header and footer within the environment
}{
\enterProblemHeader{\homeworkProblemName} % Header and footer after the environment
}
   
%----------------------------------------------------------------------------------------
%	NAME AND CLASS SECTION
%----------------------------------------------------------------------------------------

\newcommand{\hmwkTitle}{Quiz 2} % Assignment title
\newcommand{\hmwkDueDate}{Aug 19,\ 2014} % Due date
\newcommand{\hmwkClass}{Numerical Linear Algebra} % Course/class
\newcommand{\hmwkClassTime}{6:00 pm} % Class/lecture time
\newcommand{\hmwkClassInstructor}{Lecture time:} % Teacher/lecturer
\newcommand{\hmwkAuthorName}{Weiyi Chen} % Your name

%----------------------------------------------------------------------------------------
%	TITLE PAGE
%----------------------------------------------------------------------------------------

\title{
\vspace{2in}
\textmd{\textbf{\hmwkClass:\ \hmwkTitle}}\\
\normalsize\vspace{0.1in}\small{Due\ on\ \hmwkDueDate}\\
\vspace{0.1in}\large{\textit{\hmwkClassInstructor\ \hmwkClassTime}}
\vspace{3in}
}

\author{\textbf{\hmwkAuthorName}}
\date{} % Insert date here if you want it to appear below your name

%----------------------------------------------------------------------------------------

\begin{document}

\maketitle

%----------------------------------------------------------------------------------------
%	TABLE OF CONTENTS
%----------------------------------------------------------------------------------------

%\setcounter{tocdepth}{1} % Uncomment this line if you don't want subsections listed in the ToC

%\newpage
%\tableofcontents
\newpage

%----------------------------------------------------------------------------------------
%   PROBLEM 1
%----------------------------------------------------------------------------------------

\begin{homeworkProblem}
    \begin{homeworkSection}{(i)}
        The payoff matrix is:
        \begin{equation}
            M = \left( \begin{array} {cccc} 
            1  & 1  & 1  & 1  \\
            32 & 38 & 42 & 44 \\
            0 & 2 & 6 & 8  \\
            8 & 2 & 0 & 0 
            \end{array} \right)
        \end{equation}
        It is easy to verify that $det(M) = 16 \neq 0$, which implies $rank(M) = 4$, therefore the four assets are non-redundant. 
    \end{homeworkSection}
    \begin{homeworkSection}{(ii)}
        Let $s_\tau$ be the price vector at time $\tau$ of the replicable derivative security, then
        \begin{equation}
            s_\tau = (0, 4, 6, 6)
        \end{equation}
        Let $\Theta$ be the positions vector of the replicating portfolio, then
        \begin{equation}
             s_\tau = \Theta^tM_\tau
        \end{equation}
        We derive the positions vector as
        \begin{equation}
            \Theta^t = s_\tau M_\tau^{-1} = (24, -0.5, 0.5, -1)^t
        \end{equation}
    \end{homeworkSection}
    \begin{homeworkSection}{(iii)}
        The price vector of the securities at time $t_0$ is
        \begin{equation}
            S_{t_0} = (1, 40, 8, 5)^t
        \end{equation}
        To decide whether this market model is arbitrage-free, we must solve the linear system $M_\tau Q = S_{t_0}$ and check whether all the entries of the vector $Q$ are positive, as stated
        \begin{equation}
            Q = M_\tau^{-1} S_{t_0} = (1, -1.5, 0.5, 1)^t
        \end{equation}
        since not all the entries of $Q$ are positive, we conclude that the one period market model is not arbitrage-free.
    \end{homeworkSection}
    \begin{homeworkSection}{(iv)}
        A possible arbitrage can be a short position of three months call on the stock with strike $36$ at value \$8. It is easy to verify the payoff is always at least $0$ for all states, since even if the underlying asset rises to the highest price \$44, the P\&L of this option is still 0. \\
        Formally, we can also apply the conclusion of state price $Q$ to continue constructing an arbitrage portfolio, let $\Theta$ be the positions vector of the arbitrage portfolio, let the payoff at $\tau$ as $V_\tau$ then
        \begin{equation}
            V_\tau = (0,1,0,0)^t
        \end{equation}
        since the second entry of $Q$ is negative. Then according to $\Theta^t M_\tau = V_\tau^t$
        \begin{equation}
            \Theta = (-36, 1, -1, 0.5)^t
        \end{equation}
        which is the argitrage portfolio generating \$1.5 at time $\tau$, therefore this is a type II arbitrage.
    \end{homeworkSection}
\end{homeworkProblem}

\end{document}
%%%%%%%%%%%%%%%%%%%%%%%%%%%%%%%%%%%%%%%%%
% Structured General Purpose Assignment
% LaTeX Template
%
% This template has been downloaded from:
% http://www.latextemplates.com
%
% Original author:
% Ted Pavlic (http://www.tedpavlic.com)
%
% Note:
% The \lipsum[#] commands throughout this template generate dummy text
% to fill the template out. These commands should all be removed when 
% writing assignment content.
%
%%%%%%%%%%%%%%%%%%%%%%%%%%%%%%%%%%%%%%%%%

%----------------------------------------------------------------------------------------
%	PACKAGES AND OTHER DOCUMENT CONFIGURATIONS
%----------------------------------------------------------------------------------------

\documentclass{article}

\usepackage{fancyhdr} % Required for custom headers
\usepackage{lastpage} % Required to determine the last page for the footer
\usepackage{extramarks} % Required for headers and footers
\usepackage{graphicx} % Required to insert images
\usepackage{lipsum} % Used for inserting dummy 'Lorem ipsum' text into the template
\usepackage{listings}
\usepackage{color}
\usepackage{amsmath}

\definecolor{dkgreen}{rgb}{0,0.6,0}
\definecolor{gray}{rgb}{0.5,0.5,0.5}
\definecolor{mauve}{rgb}{0.58,0,0.82}

\lstset{frame=tb,
  language=Python,
  aboveskip=3mm,
  belowskip=3mm,
  showstringspaces=false,
  columns=flexible,
  basicstyle={\small\ttfamily},
  numbers=none,
  numberstyle=\tiny\color{gray},
  keywordstyle=\color{blue},
  commentstyle=\color{dkgreen},
  stringstyle=\color{mauve},
  breaklines=true,
  breakatwhitespace=true
  tabsize=3
}

% Margins
\topmargin=-0.45in
\evensidemargin=0in
\oddsidemargin=0in
\textwidth=6.5in
\textheight=9.0in
\headsep=0.25in 

\linespread{1.1} % Line spacing

% Set up the header and footer
\pagestyle{fancy}
\lhead{\hmwkAuthorName} % Top left header
\chead{\hmwkClass\ (\hmwkClassInstructor\ \hmwkClassTime): \hmwkTitle} % Top center header
\rhead{\firstxmark} % Top right header
\lfoot{\lastxmark} % Bottom left footer
\cfoot{} % Bottom center footer
\rfoot{Page\ \thepage\ of\ \pageref{LastPage}} % Bottom right footer
\renewcommand\headrulewidth{0.4pt} % Size of the header rule
\renewcommand\footrulewidth{0.4pt} % Size of the footer rule

\setlength\parindent{0pt} % Removes all indentation from paragraphs

%----------------------------------------------------------------------------------------
%	DOCUMENT STRUCTURE COMMANDS
%	Skip this unless you know what you're doing
%----------------------------------------------------------------------------------------

% Header and footer for when a page split occurs within a problem environment
\newcommand{\enterProblemHeader}[1]{
\nobreak\extramarks{#1}{#1 continued on next page\ldots}\nobreak
\nobreak\extramarks{#1 (continued)}{#1 continued on next page\ldots}\nobreak
}

% Header and footer for when a page split occurs between problem environments
\newcommand{\exitProblemHeader}[1]{
\nobreak\extramarks{#1 (continued)}{#1 continued on next page\ldots}\nobreak
\nobreak\extramarks{#1}{}\nobreak
}

\setcounter{secnumdepth}{0} % Removes default section numbers
\newcounter{homeworkProblemCounter} % Creates a counter to keep track of the number of problems

\newcommand{\homeworkProblemName}{}
\newenvironment{homeworkProblem}[1][Problem \arabic{homeworkProblemCounter}]{ % Makes a new environment called homeworkProblem which takes 1 argument (custom name) but the default is "Problem #"
\stepcounter{homeworkProblemCounter} % Increase counter for number of problems
\renewcommand{\homeworkProblemName}{#1} % Assign \homeworkProblemName the name of the problem
\section{\homeworkProblemName} % Make a section in the document with the custom problem count
\enterProblemHeader{\homeworkProblemName} % Header and footer within the environment
}{
\exitProblemHeader{\homeworkProblemName} % Header and footer after the environment
}

\newcommand{\problemAnswer}[1]{ % Defines the problem answer command with the content as the only argument
\noindent\framebox[\columnwidth][c]{\begin{minipage}{0.98\columnwidth}#1\end{minipage}} % Makes the box around the problem answer and puts the content inside
}

\newcommand{\homeworkSectionName}{}
\newenvironment{homeworkSection}[1]{ % New environment for sections within homework problems, takes 1 argument - the name of the section
\renewcommand{\homeworkSectionName}{#1} % Assign \homeworkSectionName to the name of the section from the environment argument
\subsection{\homeworkSectionName} % Make a subsection with the custom name of the subsection
\enterProblemHeader{\homeworkProblemName\ [\homeworkSectionName]} % Header and footer within the environment
}{
\enterProblemHeader{\homeworkProblemName} % Header and footer after the environment
}
   
%----------------------------------------------------------------------------------------
%	NAME AND CLASS SECTION
%----------------------------------------------------------------------------------------

\newcommand{\hmwkTitle}{Quiz 3} % Assignment title
\newcommand{\hmwkDueDate}{Aug 20,\ 2014} % Due date
\newcommand{\hmwkClass}{Numerical Linear Algebra} % Course/class
\newcommand{\hmwkClassTime}{6:00 pm} % Class/lecture time
\newcommand{\hmwkClassInstructor}{Lecture time:} % Teacher/lecturer
\newcommand{\hmwkAuthorName}{Weiyi Chen} % Your name

%----------------------------------------------------------------------------------------
%	TITLE PAGE
%----------------------------------------------------------------------------------------

\title{
\vspace{2in}
\textmd{\textbf{\hmwkClass:\ \hmwkTitle}}\\
\normalsize\vspace{0.1in}\small{Due\ on\ \hmwkDueDate}\\
\vspace{0.1in}\large{\textit{\hmwkClassInstructor\ \hmwkClassTime}}
\vspace{3in}
}

\author{\textbf{\hmwkAuthorName}}
\date{} % Insert date here if you want it to appear below your name

%----------------------------------------------------------------------------------------

\begin{document}

\maketitle

%----------------------------------------------------------------------------------------
%	TABLE OF CONTENTS
%----------------------------------------------------------------------------------------

%\setcounter{tocdepth}{1} % Uncomment this line if you don't want subsections listed in the ToC

%\newpage
%\tableofcontents
\newpage

%----------------------------------------------------------------------------------------
%   PROBLEM 1
%----------------------------------------------------------------------------------------

\begin{homeworkProblem}
  Since $Q$ is an orthogonal matrix, then $Q^tQ = I$ or $Q^t = Q^{-1}$, therefore the characteristic polynomial of $Q^tAQ$ is
  \begin{align}
    P_{Q^tAQ}(x) &= det(\lambda I - Q^tAQ) \\
    &= det(Q^t\lambda IQ - Q^tAQ) \\
    &= det(Q^t (\lambda I - A)Q) \\
    &= det(Q^t) det(\lambda I - A) det(Q) \\
    &= det(Q^{-1})det(Q)  det(\lambda I - A) \\
    &= det(\lambda I - A) = P_{A}(x)
  \end{align}
  which is the characteristic polynomial of $A$.
\end{homeworkProblem}

%----------------------------------------------------------------------------------------
%   PROBLEM 2
%----------------------------------------------------------------------------------------

\begin{homeworkProblem}
  $A$ is positive definite if and only if all of its $n$ leading principal minors are strictly positive. There are four leading principal minors, one of order 1:
  \begin{equation}
    \left| a_{11} \right| = 1 > 0
  \end{equation}
  one of order 2:
  \begin{equation}
    \left| \begin{array} {cc} 
    a_{11} & a_{12} \\
    a_{21} & a_{22} 
    \end{array} \right| = 0.96 > 0
  \end{equation}
  one of order 3:
  \begin{equation}
    \left| \begin{array} {ccc} 
    a_{11} & a_{12} & a_{13}\\
    a_{21} & a_{22} & a_{23}\\
    a_{31} & a_{32} & a_{33} 
    \end{array} \right| = 0.8775 > 0
  \end{equation}
  one of order 4:
  \begin{equation}
    \left| A \right| = 0.853275 > 0
  \end{equation}
  Therefore the matrix $A$ is symmetric positive definite.
\end{homeworkProblem}

%----------------------------------------------------------------------------------------
%   PROBLEM 3
%----------------------------------------------------------------------------------------

\begin{homeworkProblem}
  According to the description of $C_N$,
  \begin{equation}
    C_N(i,j) = \begin{cases}
      1, &\text{ if }i = j\\
      -1, &\text{ if }i = j-1\\
      0, &\text{ otherwise }
    \end{cases}
  \end{equation}
  for $1 \le i \le N, 1 \le j \le N+1$. \\
  For its transpose, $C_N^t(i,j) = C_N(j,i)$ for any $1 \le i \le N+1, 1 \le j \le N$. Therefore,
  \begin{equation}
    C_NC_N^t(i,j) = \sum_k C_N(i,k) C_N^t(k,j) = \sum_k C_N(i,k)C_N(j,k)
  \end{equation}
  In case $i=j$,
  \begin{equation}
    C_NC_N^t(i,j) = \sum_k C_N^2(i,k) = C_N^2(i,i) + C_N^2(i,i+1) = 2
  \end{equation}
  In case $i=j+1$,
  \begin{equation}
    C_NC_N^t(i,j) = \sum_k C_N(i,k)C_N(i-1,k) = C_N(i,i)C_N(i-1,i) = -1
  \end{equation}
  In case $i=j-1$,
  \begin{equation}
    C_NC_N^t(i,j) = \sum_k C_N(i,k)C_N(i+1,k) = C_N(i,i+1)C_N(i+1,i+1) = -1
  \end{equation}
  Otherwise, $C_NC_N^t(i,j) = 0$ since the inner product (i.e., $\sum_k C_N(i,k)C_N(j,k)$) of any two rows in matrix $C_N$ with index distance greater than $1$ is $0$. Given the entries above, it's exactly $B_N$ given in the condition, so
  \begin{equation}
    B_N = C_N C_N^t
  \end{equation}
  $C_N$ is not the Choleskey decomposition of the matrix $B_N$ because the Cholesky decomposition is a decomposition of a Hermitian, positive-definite matrix into the product of a lower triangular matrix and its conjugate transpose, where the lower triangular matrix must be a square matrix while $C_N$ is not.
\end{homeworkProblem}

%----------------------------------------------------------------------------------------
%   PROBLEM 4
%----------------------------------------------------------------------------------------

\begin{homeworkProblem}
  \begin{homeworkSection}{(i)}
    A symmetric matrix $A$ is positive definite if and only if there exists a nonsingular matrix $B$ such that $A=B^tB$, then for any column vector $v \neq 0$, we have
    \begin{equation}
      v^tM^tAMv=v^tM^tB^tBMv=v^t(BM)^t(BM)v=(BMv)^t\cdot(BMv)\ge0
    \end{equation}
    therefore $M^tAM$ is semipositive definite.
    In addition, $M^tAM$ is symmetric because
    \begin{equation}
      (M^tAM)^t = M^t A^t M = M^t A M
    \end{equation}
    where $A$ is given symmetric, i.e. $A^t = A$. 
    In conclude, $M^tAM$ is symmetric semipositive definite.
  \end{homeworkSection}
  \begin{homeworkSection}{(ii)}
    If the columns of the matrix $M$ is linearly independent, then $M$ is nonsingular (i.e. $det(M)\neq0$). $BM$ is also nonsingular, where $B$ is defined in part(i), because
    \begin{equation}
      det(BM) = det(B)det(M) \neq 0
    \end{equation}
    which implies that the columns of $BM$ are linearly independent. We re-apply the formula in part(i) and find that, for any column vector $v \neq 0$,
    \begin{equation}
      v^tM^tAMv = (BMv)^t\cdot(BMv) > 0
    \end{equation}
    since any linear combination (i.e. $v$) of columns of $BM$, forming $BMv$ cannot be a zero vector. In conclude, the matrix $M^tAM$ is symmetric positive definite from the formula above. (Symmetry proof is the same as part(i), $(M^tAM)^t = M^t A^t M = M^t A M$.) \\
    On the other hand, if the matrix $M^tAM$ is symmetric positive definite, then $M^tAM$ is nonsingular. Assume by contradiction that the columns of the matrix $M$ are not linearly independent (i.e.,$det(M) = det(M^t) = 0$), then
    \begin{equation}
      det(M^tAM) = det(M^t)det(A)det(M) = 0
    \end{equation}
    which contradicts to the fact that $M^tAM$ is nonsingular. So the columns of the matrix $M$ are linearly independent.
  \end{homeworkSection}
\end{homeworkProblem}

%----------------------------------------------------------------------------------------
%   PROBLEM 5
%----------------------------------------------------------------------------------------

\begin{homeworkProblem}
  A matrix can be a correlation matrix only if it is symmetric semidefinite with all the entries at the diagonal as $1$. It is easy to verify the matrix given (denote as $A$) is symmetric with all diagonal entries being $1$, for the semidefinite condition, we can check whether all its principle minors are $0$:\\
  3 of order 1:
  \begin{equation}
    \left| a_{11} \right| = \left| a_{22} \right| = \left| a_{33} \right| = 1 > 0
  \end{equation}
  3 of order 2:
  \begin{equation}
    \left| \begin{array} {cc} 
    a_{11} & a_{12} \\
    a_{21} & a_{22} 
    \end{array} \right| = 0.99 > 0,
    \left| \begin{array} {cc} 
    a_{11} & a_{13} \\
    a_{31} & a_{33} 
    \end{array} \right| = 0.96 > 0
    \left| \begin{array} {cc} 
    a_{22} & a_{23} \\
    a_{32} & a_{33} 
    \end{array} \right| = 0.91 > 0
  \end{equation}
  1 of order 3:
  \begin{equation}
    \left| A \right| = 0.848 > 0
  \end{equation}
  Therefore it is a correlation matrix. \\
  The Cholesky factor of the matrix $A = LL^t$ is 
  \begin{equation}
    L = \left( \begin{array} {ccc}
    1 & 0 & 0 \\
    0.1 & 0.99498744 & 0 \\
    0.2 & -0.3216121 & 0.92550832
    \end{array} \right),
    L^t = \left( \begin{array} {ccc}
    1 & 0.1 & 0.2 \\
    0 & 0.99498744 & -0.3216121 \\
    0 & 0 & 0.92550832
    \end{array} \right)   
  \end{equation}
  using the formula
  \begin{equation}
    L_{kk} = \sqrt{a_{kk} - \sum_{j=1}^{k-1} L_{kj}^2}, L_{ik} = \frac{1}{L_{kk}} \left ( a_{ik} - \sum_{j=1}^{k-1} L_{ij}L_{kj} \right )
  \end{equation}
\end{homeworkProblem}
\end{document}